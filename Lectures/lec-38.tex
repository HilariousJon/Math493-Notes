\begin{section}{Induced Representation}
  We can easily derived a representation of a subgroup for a given group by directly taking restriction on the representation. However, it is relatively unknown for us to whether be able to derived a representation of a subgroup into a representation of the group. The goal is to construct a functor in the opposite direction of the restriction functor, in particular, it will be the \textbf{Induced Functor}. The representation derived will be \textbf{Induced Representation}. The general intuition is that we have a group \( G \) and a subgroup \( H \), we want to relate the representation of \( G \) and the representation of \( H \), in particular their irreducible representation. Notably, a induced representation of a irreducible representation \textbf{may not} be irreducible, and the restriction of a irreducible representation \textbf{also may not} be irreducible.
  
  First introduce the setup: Let \( G \) be finite group, \( H\leq G \) be a subgroup of \( G \), we then have two categories: \textbf{The category of \( \mb G \)-repr and the catgeory of \( \mb H \)-repr.} We already gave the \textbf{restriction functor} \( \rho \rsa \rho \circ \text{inclusion map} \).
  \begin{notation}
    We denote the restrction functor for a representation as \( \rho |_H \) or \( \Res_{H}^G(\rho) \).
  \end{notation}

  \begin{definition}[\textbf{Induced Functor}]
    \label{def:induce}
    In particular, a \underline{induced functor} means: Given a \( H \)-repr \( W \), we construct a \( G \)-repr \( V \) together with a \textbf{morphism} of \underline{\( H \)-repr} \( f: W \to V \), with the following \textbf{universal property}:

    Given any \( G \)-repr \( V' \) and morphism of \( H \)-repr \( f': W \to V' \), there exists a \textbf{unique morphism of \( \mb G \)-repr} \( g: V \to V' \), s.t. the following diagram is commutative:
    \[
      \begin{tikzcd}
      W \arrow[r, "f"] \arrow[d, "f'"'] & V \arrow[d, "g", ld, dashed] \\
      V' &  
      \end{tikzcd}
    \]

    Note that here \( V, V' \) are both \( G \)-repr. In particular we see: \( g\circ f = f' \).

    Equivalently, it means:
    \[
      \begin{aligned}
        \Hom_{G-\text{repr}} (V, V') & \to \Hom_{H-\text{repr}}(W, V') \\ 
        g &\rsa g\circ f 
      \end{aligned}
    \] 

    is a \textbf{bijection} for all \( G \)-repr \( V' \).
  \end{definition}
  Before the construction, we shall first give a warm-up proposition.

  \begin{proposition}
    If every irreducible repr. of \( H \) has dimension \( \leq d \), then every irred. repr. of \( G \) has dimension \( \leq d\cdot (G:H) \).

    In particular, if \( H \) is abelian, then every irreducible repr of \( G \) has dimension \( \leq (G:H) \).
  \end{proposition}
  
  \begin{proof}
    Let \( V \ne 0 \) be any repr. of \( G \), we can also view it as a repr of \( H \) by \( H \xrightarrow{\text{incl}} G \to \GL(V) \). Suppose \( W \subseteq V \) is an irred. \( H \)-repr (we can always find such a subrepr by \textbf{Theorem} \ref{thm:complement}), then \( \dim W \leq d \) by assumption.

    We now want to enlarge \( W \) into a \( G \)-subrepr of \( V \). Suppose \( \sigma \in \bace{\qo{G}{H}}_l \), Define: \( W_\sigma := gW = \rho(g)(W)\subseteq V \) where \( \sigma = gH \) for some \( g\in G \). One can check that this does not depends on the choice of \( g \) by definition of cosets.

    Let \( V' = \sum_{\sigma \in \bace{\qo{G}{H}}_l} W_\sigma \subseteq V \) be a \( G \)-subrepr of \( V \), indeed since \( g' W_\sigma \subseteq W_{g'\sigma} \) (note here the notation using the action of \( G \) on \( \bace{\qo{G}{H}}_l \)).

    Note that each \( W_\sigma \cong \; W \) as  vector spaces via \( W \xrightarrow{\rho(g)} W_\sigma \) where \( \sigma = gH \). Thus \( \dim(W_\sigma) = \dim W \; \forall \; \sigma \). 
    \[
      \implies \dim(V') \leq \bace{\dim W} \cdot (G:H) \leq d \cdot (G:H)
    \] 

    If \( V \) is irred. \( G \)-repr, then \( V' = V \implies \dim V \leq d \cdot (G:H)\).
  \end{proof}

  We shall use \textbf{similar} construction method for the construction of induced functor.

  Fix an \( H \)-repr \( W \), choose \( g_1, \ldots, g_r \in G \), s.t. \( \bace{\qo{G}{H}}_l = \{g_1 H, \ldots, g_r H\} \), s.t. \( g_1 = e \).

  For each \( i \in \br{1,r} \), choose a vector subspace \( W_i \cong \; W \) with an isomorphism:
  \[
    \begin{aligned}
      W & \xrightarrow{\sim} W_i \\ 
      w &\mapsto g_i w \text{ (notation for the map.)}
    \end{aligned}
  \] 

  Let \( V:= W_1 \oplus \cdots \oplus W_r \), shall define an action of \( G \) on \( V \): Define \( (g, (w_1, \ldots, w_r)) \) as \( gw_1 + \cdots + gw_r \) where \( gw_i \) is defined as follows:

  If \( w_i = g_i v_i, \; v_i \in W \), there exists a \textbf{unique} \( j \), s.. \( gg_i H = g_j H \), thus define \( gw_i := g_j(hv_i) \in W_j \). We want to see \( V \) gives a representation of \( G \). Let's check that it gives an action of \( G \) on \( V \) and its linearity:
  \begin{enumerate}
    \item \( V \) gives a \textbf{group action}:
      \begin{itemize}
        \item If \( g=e \), then \( j=i, h=e \in H \) thus \( g(g_i v_i) = g_i v_i \), yields the results.
        \item Suppose \( g,g' \in G \), \( w_i = g_i v_i \in W_i \), we need to check:
          \[
            g'(gw_i) = (g'g)w_i
          \]

          Let \( j,h \) as before: \( gg_i = g_j h \) and \( gw_i = g_j(hv_i) \), let \( k \), s.t. \( g'g_j \in g_k H \implies g'g_j = g_kh', \; h'\in H \). Thus:
          \[
            \begin{aligned}
              g'(gw_i) &= g'(gg_i v_i) \\ 
              &= g'(g_j (hv_i)) \\ 
              &= g_k(\underbrace{h'(hv_i)}_{H-\text{repr}}) \\ 
              &= g_k((h'h)v_i) \\ 
              (g'g)g_i &= g'(gg_i) \\ 
                       &= g'(g_j h) \\ 
                       &=(g_k h')h \\ 
                       &= g_k(h'h) \\ 
                      \implies (g'g)w_i &= g_k((h'h)v_i) 
                      \iff (g'g)w_i &= g'(gw_i)
            \end{aligned}
          \] 

          So it is indeed an action on \( \mb{\bm{\bigcup_{i=1}^r} W_i} \). To further see it is also a \( G \)-action on \( V \), we need to check \textbf{linearity}, too.
      \end{itemize}
    \item For all \( g\in G \), to see \( V \to V, \; x \mapsto gx \) is linear, it is enough to check its restriction on each \( W_i \) is linear, i.e. \( \bm{\bigcup_{i=1}^r W_i} \).

      On \( W_i \): if \( gg_i = g_j h \; h \in H \), the map is given by:
      \[
        \begin{tikzcd}
        W_i & W \arrow[l, "\sim"'] \arrow[d, "h \text{ linear}"] \\
        W_j & W \arrow[l, "\sim"']
        \end{tikzcd}
      \]
      
      thus the induced map \( W_i \to W_j \) is also linear, which yields the result.

      Note here we have a map \textbf{canonically}:
      \[
        \begin{aligned}
          W & \xrightarrow[g_i]{\sim} W_i \subseteq V \\ 
        \end{aligned}
      \] 

      where \( g_i \) as we defined before. 

      Also note:
      \begin{equation}
        \label{eq:canonical}
        \begin{aligned}
         f_W: W &\xrightarrow{\sim} W_1 \subseteq V \\ 
          w &\rsa e_w \quad \text{ (since $g_1 = e$)}
        \end{aligned}
      \end{equation}

      such map is linear and \textbf{morphisms of \( \mb H \)-repr}.
      \begin{remark}
        If we have an element in \( W_i \) of the form \( g_i v \), this is actually \( g_i(ev) \).
      \end{remark}
  \end{enumerate}

  \begin{eg}
    Suppose \( W \) is the trivial \( 1 \)-dimensional representation of \( H \), then what is \( V = \Ind_{H}^{G}(W)\)? 

    Note that \( V \) has a basis indexed by \( \bace{\qo{G}{H}}_l \), it will be the permutation representation of \( G \) to the action of \( G \) acting on \( \bace{\qo{G}{H}}_l \) be left multiplication.
  \end{eg}

  \begin{proof}[\textbf{Proof of Universal Property}]
    We want to show that any \textbf{morphisms} of \( G \)-repr that makes the diagram in the definition commutative should acting on the same on those ``generators" \( W_i \), in particular: if \( g|_{W_i} \) maps \( g_i v_i \) to \( g_if'(v_i) \) will yeilds the \textbf{uniqueness}.
    \begin{itemize}
      \item \textbf{Existence}: Define \( g: V\to V' \) by define it on each \( W_i \): \( g(g_iv)  = g_i f'(v) \; v \in W \). This is clearly linear and clear that \( g\circ f = f' \). 
      \item Left to check by our definition it is a \textbf{morphism} of \( G \)-repr: which is to check:
        \[
          g(a(g_i v)) = ag(g_i v)  \qquad \forall \; a \in G 
        \] 

        See that:
        \[
          \begin{aligned}
            g(a(g_i v)) &= g(g_j(hv)) = g_j(f'(hv)) \\ 
                        &= g_j(hf'(v)) \quad \text{since $f'$ is morphism of $H$-repr.} \\ 
                        &= g_j(hf'(v)) \\ 
                        &= (g_jh)f'(v) \\ 
                        &= (ag_i)f'(v) \\ 
                        &= a(g_if'(v)) = ag(g_i(v))
          \end{aligned}
        \] 
    \end{itemize}
  \end{proof}

  \begin{note}
    \leavevmode 
    \begin{itemize}
      \item The universal properties implies that the construction is independent of the choice of \( g_1, \ldots, g_r \) up to a canonical isomorphism. (Only depends on the objects in the set of the left cosets of \( H \)).
      \item Another consequenc of the universal property is that we can extend \( \Ind_{H}^G(-) \) to a functor from the catgeory of \( H \)-repr to category of \( G \)-repr.

        Suppose \( \vp: W \to W' \) be morphism of \( H \)-repr, then the following diagram is commutative:
        \[
        \begin{tikzcd}[row sep=large, column sep=huge]
          W \arrow[r, "f_W"] \arrow[d, "\varphi"'] & \text{Ind}_H^G(W) \arrow[d, dashed, "\Ind_{H}^G (\vp)"] \\
          W' \arrow[r, "f_{W'}"']                   & \text{Ind}_H^G(W')
        \end{tikzcd}
        \] 

        where \( f_W \) and \( f_W' \) is defined as the canonical map as in \textbf{Def} \ref{eq:canonical}. By the universal property, \( \Ind_{H}^G(\vp) \) is the unique morphism of \( G \)-repr that makes the diagram commutative, and leads to the functorial property of \( \Ind_{H}^G(-) \).
    \end{itemize}
  \end{note}

  \begin{notation}
    Notation for \( V \) will be \( \Ind_{H}^{G}(W) \).
  \end{notation}
  
  \begin{subsection}{Character of an Induced Representation}
    In this section, we shall introduce the character for the induced representation.
    \begin{note}
      If \( \rho': G \to \GL(V) \) repr of \( G \), then 
      \[
        \chi_{\rho'|_H} = \chi_{\rho'}|_H
      \] 
    \end{note}


    Now give \( \rho: H \to \GL(W) \), lets compute \( \chi_{\Ind_{H}^G}(\rho) \).

    Recall the definition of \( \Ind_{H}^G(\rho) \), we choose \( g_1,\ldots, g_r \), s.t. \( g_1 = e \) and \( \bace{\qo{G}{H}}_l = \{g_1H ,\ldots, g_rH\} \). See that
    \[
      V = \bigoplus_{i=1}^r g_i W \qquad W \cong\; g_i W 
    \] 

    the action of \( G \) is given by \( g(g_iw) = g_j(hw) \) for \( w \in W \) where \( gg_i = g_j h \) for some \( j \) and \( h\in H \). Thus 
    \[
      \chi_{\Ind_H^G(\rho)}(g) = \sum_{i, g^{-1}_i g g_i \in H} \chi_{W} (g^{-1}_i g g_i)
    \] 

    where we compute this using a basis \( e_1, \ldots, e_n \) for \( W \) and the basis \( (g_i e_k)_{i \in \br{1,r}, k\in \br{1,n}} \) for \( V \). The idea here is we only consider those \( i=j \) as they will be block matrix on the diagonal, while those \( i\ne j \) will not contribute to the calculation of trace.
    \begin{note}
      If \( i=j \), then the diagram is commutative:
    \[
      \begin{tikzcd}[row sep=large, column sep=huge]
          \sigma_i W \arrow[r] \arrow[d, "\cong"'] & \sigma_i W \arrow[d, "\cong"] \\
          W \arrow[r, "\rho(g_i^{-1} g g_i)"] & W
      \end{tikzcd}
    \]  
    \end{note}

    \begin{theorem}[\textbf{Frobenius Reciprocity}]
        If \( \psi \in \cC(H) \) and \( \vp \in cC(G) \), then:
        \[
          \angl{\Ind_H^G(\psi), \vp}_{\cC(G)} = \angl{\psi, \Res_H^G(\vp)}_{\cC(H)}
        \] 
    \end{theorem}

    \begin{question}
      Why this theorem so important?

      This gives us a easy way to calculate and understand the irreducible decomposition of induced representation.
    \end{question}

    \begin{proof}
      Since \( \cC(G) \) has a basis given by character of irred. repr. of \( G \) and similarly for \( H \), using the conjugate bilinearity of the Hermitian product, it is enough to check the formula when \( \vp = \chi_V, \psi = \chi_W \) where \( V,W \) are irred. \( G,H \)-repr respectively.

      Hence \( \Res(\vp) = \chi_{\Res(V)}, \Ind(\psi) = \chi_{\Ind(W)} \), hence we need to see:

      \[
      \begin{aligned}
        \langle \chi_{\text{Ind}(W)}, \chi_V \rangle_{\mathcal{C}(G)} &= \langle \chi_W, \chi_{\text{Res}(V)} \rangle_{\mathcal{C}(H)} \\
        || \quad \quad \quad & \quad \quad \quad || \\
        \# \text{ copies of } V \text{ in } \text{Ind}(W) & \quad \# \text{ copies of } W \text{ in } \text{Res}(V) \\
        || \quad \quad \quad  & \quad \quad \quad || \; \longleftarrow \text{ \textbf{Schur's Lemma} \ref{lem:schur}}\\
        \dim_{\mathbb{C}} \text{Hom}_{G\text{-repr}}(\text{Ind}(W), V) & \quad \dim_{\mathbb{C}} \text{Hom}_{H\text{-repr}}(W, \text{Res}(V))
      \end{aligned}
      \] 

      where the last layer of equation can be refered to \textbf{Proposition} \ref{prop:copy}.

      But the universal property \ref{def:induce} of \( \Ind(W) \) shows that:
      \[
      \begin{tikzcd}
        W \arrow[r] \arrow[d] &  \text{Ind}(W) \arrow[ld] \\
                              V &                         
      \end{tikzcd}
      \] 

      \[
        \implies \Hom_{G-\text{repr}}(\Ind(W), V) \xlongleftrightarrow{\text{bij}} \Hom_{H-\text{repr}} (W,V) = \Hom_{H-\text{repr}}(W, \Res(V))
      \] 
      
      This is a isomorphism of \( \CC \)-vector spaces, thus the two dimensions are equal
    \end{proof}
  \end{subsection}
\end{section}
