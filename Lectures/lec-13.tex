\begin{section}{Orbits and Orbits-Stablizer Theorem}
  \begin{definition}
    Write \( x\sim y \) for \( x,y \; \in \; X \), if \( \exists \; g \; \in G \), s.t. \( gx = y \).
  \end{definition}
  \begin{lemma}
    Such gives us a equivalent relation, directly check by \textbf{reflexive, symmetric, transitive}.
  \end{lemma}

  \begin{conclusion}
    We get a partition of \( X \) into equivalence classes, called \textbf{orbits}. If \( x \in X \), then the corresponding equivalence classes is given by:
    \[
      \{gx \; | \; g \in G\}
    \] 

    which is denoted by \( \mathbf{Gx} \) or \( \mathbf{O(x)} \).

    \begin{notation}
      \( X / G \) denotes the sets of the orbits of \( X \).
    \end{notation}
  \end{conclusion}

  \begin{definition}
    The action of \( G \) on \( X \) is \underline{transitive} if \( X \) has only one orbits, which is:
    \[
      \forall \; x,y \in X, \; \exists \; g \in G \; s.t. \; gx = y
    \] 

    \begin{eg}
      The action given by the left multiplication of \( G \) on itself is \textbf{transitive}.
    \end{eg}

    \begin{eg}
      Induced by above example, the action of \( G \) on the \textbf{set of left cosets} of \( H \) is also transitive.
    \end{eg}
  \end{definition}

  \begin{definition}
    For every \( x \in X \), the \underline{stablizer} of \( x \in G \) is given by:
    \[
      Stab_G(x) = \{g \in G \; | \; gx = x\}
    \] 

    namely those elements in \( G \) that doesn't move the position of \( x \).
  \end{definition}

  \begin{lemma}
    \( Stab_G(x) \leq G \) being a subgroup.
  \end{lemma}

  \begin{eg}
    Consider the action of \( G \) on itself by conjugation, the orbits of \( a \in G \) is called the \underline{conjugate class of \( a \)}. Two elements of \( G \) are conjugate of each other if they lie in the same conjugate class (\textbf{same orbit}).

    What is the stablizer in this case?
    \[
      Stab_G(x) = \{y \in G \; | \; yxy^{-1} = x\} =: C_G(x)
    \] 

    which is the \underline{centralizer} of \( x \) in \( G \).

    \begin{note}
      \( C_G(x) = G \) iff \( x \in Z(G) \)
    \end{note}
  \end{eg}

  \begin{remark}
    Consider the conjugacy classes of \( S_n \), then \( \sigma, \tau \in S_n \) are conjugate of each other if and only if when they written as \textbf{product of disjoint cycles}, then \# of \( k \)-cycle for both of them is the \textbf{same} for all \( k \) (\textbf{they have same cycle type}).
  \end{remark}

  We now introduce Orbit-Stablizer theorem.
  \begin{theorem}
    \label{orbi-stab}
    If \( G \) acts on \( X \), then for every \( x \; \in \; X \):
    \[
      \# O(x) = (G : Stab_G(x))
    \] 

    In particular, if \( G \) is finite group, then
    \[
      \# O(x) \; \mid  \; | G | 
    \] 
  \end{theorem}

  \begin{proof}
    Define a map:
    \[
      \begin{aligned}
        f: (G / Stab_G(x))_l & \longrightarrow O(x) \\ 
        f(g Stab_G(x)) &= gx 
      \end{aligned}
    \] 

    \begin{itemize}
      \item Well-defineness + injectivity:
        \[
          \begin{aligned}
            g_1 Stab_G(x) &= g_2 Stab_G (x) \\ 
            \iff g_2^{-1} g_1 &\in Stab_G(x) \\ 
            \iff (g_2^{-1} g_1) x &= x \\ 
            \iff g_2^{-1} (g_1 x) &= x \\
            \iff g_1 x &= g_2 x \\ 
          \end{aligned}
        \] 
      \item Surjectivity:
        \[
          \begin{aligned}
            y & \in O(x) \\ 
            \implies y &= gx \\ 
            \implies y &= f(g Stab_G(x))
          \end{aligned}
        \] 
    \end{itemize}

    So we see \( f \) is a bijection. The first isomorphism theorem direct yields the result.
  \end{proof}

  \begin{note}
    When the action is transitive, with \( X \) being finite, we have:
    \[
      \# O(x) = (G : Stab_G(x)) = | X |
    \] 

    In particular, transitive means there is only one orbits.
  \end{note}

  \begin{proposition}
    If \( G \) acts on \( X \), then:
    \[
      | X | = \sum_{i \in I} (G : Stab_G(x))
    \] 

    where \( x_i \) are a system of representative for the orbits of \( G \) in \( X \).
  \end{proposition}

  \begin{proof}
    We have a partition:
    \[
      X = \bigsqcup_{i \in I} O(x) \implies \# X = \sum \# O(x_i)
    \] 

    with:
    \[
      \# O(x_i) = (G : Stab_G (x_i))
    \] 
  \end{proof}

  \begin{eg}{(\textbf{Class Equation: A important special case})}
    Consider the action of \( G \times G \to G \) by \textbf{conjugation}, with \( G \) be finite group:
    \[
      \begin{aligned}
        | G | &= \sum_{i \in I} (G : C_G(x_i)) \\ 
        \implies | G | &= | Z(G) | + \sum_{i \in I'} (G : C_G(x_i))  
      \end{aligned}
    \] 

    where \( I' \) runs over indices such that \( (G : C_G(x_i)) > 1 \).

    Such results direct yields from the fact that elements in \( Z(G) \) attains its centralizer (stablizer) to be the whole group.

    \begin{note}
      Such only works for actions by \textbf{conjugation!}
    \end{note}
  \end{eg}

  We now give an application for orbits-stablizer theorem, which is important when we study the construction of groups.
  \begin{definition}
    If \( p \) be prime number, and \( G \) be group, if \( |G| = p^n \) for some \( n\geq 1 \), then we say \( G \) is a \underline{\( p \)-group}.
  \end{definition}
  
  \begin{proposition}
    If \( G \) is a \( p \)-group, then:
    \[
      Z(G) \neq \{e\}
    \] 
  \end{proposition}

  \begin{proof}
    Since \( p \mid | G | \) and \( (G : Stab_G(x)) \mid | G | = p^n\), then:
    \[
      p \mid (G : C_G(x_i))
    \] 
    
    whenever this is \( >1 \), then class equation yields that:
    \[
      p \mid | Z (G) |
    \] 
  \end{proof}

  \begin{corollary}
    If \( p \) prime, \( | G | = p^2 \), then \( G \) is an abelian group.
    \label{cor-1}
  \end{corollary}
  
  To proof this we need first a lemma:
  \begin{lemma}
    For all group \( G \), if \( G / Z(G) \) is cyclic, then \( G \) is abelian.
    \label{lem-1}
  \end{lemma}

  \begin{proof}
    By cyclic property, first suppose that \( x Z(G) \) is a generator of \( G / Z(G) \), then there exists \( i, j \; \in \mathbb Z \), such that:
    \[
      \begin{aligned}
        a Z(G) &= x^i Z(G) \\ 
        b Z(G) &= x^j Z(G) \\ 
        \iff a &= x^i a' \\ 
        b &= x^j b' \qquad \text{for some $a'b' \; \in Z(G)$} \\
        \implies ab &= x^i a' x^j b' = x^{i+j}a'b' \\ 
        ba &= x^j b' x^i a' = x^{i+j}b'a' \\ 
        \implies ab &= ba \qquad \text{since $a',b' \; \in Z(G)$}
      \end{aligned}
    \] 
  \end{proof}
  
  We now give proof to \textbf{Corollary} \ref{cor-1}:
  \begin{proof}
    We know that:
    \[
      Z(G) \neq \{e\}
    \] 

    Then by \textbf{Lagrange's theorem}, see that either \( Z(G) = G \) or \( | Z(G) | = p \).

    If \( Z(G) = G \), we already done! Now suppose that \( |Z(G)| = p\), now observe that every subgroup of \( Z(G) \) is a \textbf{normal subgroup} of \( G \) by definition.

    In particular, we then consider the group \( G / Z(G) \), this is of order \( p \), so it is cyclic. Then by \textbf{Lemma} \ref{lem-1}, see that \( G \) is abelian, contradict to the fact that \( Z(G) \neq G \) $\lightning$.
  \end{proof}

\end{section}
