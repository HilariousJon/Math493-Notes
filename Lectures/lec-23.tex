\begin{subsection}{Nilpotent Group}
  \begin{definition}{\textbf{(Nilpotent Group)}}
    Let \( G \) be any group, we define a sequence of subgroups \( \{Z_i(G)\}_{i\geq0} \) as follows:
    \begin{itemize}
      \item \( Z_0(G) = \{e\} \) 
      \item \( Z_1(G) = Z(G) \) 
      \item If \( Z_i(G) \leq G \) is constructed, consider:
        \[
          Z\left(\qo{G}{Z_i(G)}\right) = \qo{Z_{i+1}(G)}{Z_i(G)} 
        \] 
        for some \( Z_{i+1}(G) \unlhd G \) and containing \( Z_i(G) \).
    \end{itemize}

    The group \( G \) is \underline{nilpotent} if there exists \( n \), s.t. \( Z_n(G) = G \).
  \end{definition}

  We then give some examples for nilpotent groups.
  \begin{eg}
    nilpotent group is solvable group, since 
    \[
      Z\left(\qo{G}{Z_i(G)}\right) = \qo{Z_{i+1}(G)}{Z_i(G)} \text{ is abelian.}
    \] 
  \end{eg}

  \begin{eg}
    abelian groups are nilpotent: \( Z(G) = G = Z_1(G) \).
  \end{eg}
  
  \begin{eg}
    \( S_3 \cong \; D_6 \) is solvable but not nilpotent, since \( Z(S_3) = \{e\} \) but \( S_3 \neq \{e\} \). 
    
    In particular, \( S_3, S_4 \) are solvable, but not nilpotent.
  \end{eg}

  \begin{proposition}
    A group \( G \) is nilpotent if and only if there exists a sequence of subgroups:
    \[
      \{e\} = G_0 \subseteq G_1 \subseteq \ldots \subseteq G_r = G 
    \] 

    s.t. 
    \begin{enumerate}
      \item \( G_i \unlhd G, \; \forall\; i \in \br{0,r-1} \).
      \item \( \qo{G_i}{G_{i+1}} \subseteq Z\left( \qo{G}{G_{i-1}}\right), \; \forall \; i \in \br{1,r} \).
    \end{enumerate}
  \end{proposition}
  
  \begin{proof}
    \textbf{Sketch of Proof}: One direction is trivial. The other direction can be treated by prove \( G_i \subseteq Z_i(G), \; \forall \; i \in \br{0,r} \) by induction on \( i \).
  \end{proof}

  \begin{proposition}
    \label{prop:nil}
    Let \( N \leq G\), then:
    \begin{enumerate}
      \item \( G \) nilpotent \( \implies N \) is nilpotent.
      \item \( N \unlhd G, \; G \) is nilpotent \( \implies \qo{G}{N} \) is nilpotent.
      \item If \( \mathbf{N \leq Z(G)} \) and \( \qo{G}{N} \) is nilpotent \( \implies G \) is nilpotent.
    \end{enumerate}

    \begin{note}
      If we just know \( N \unlhd G \) nilpotent and \( \qo{G}{N} \) nilpotent \( \not\implies G\)  nilpotent.
    \end{note}
  \end{proposition}
  
  \begin{proof}
    Almost the same as what we have done for solvable group in \textbf{Propositpion} \ref{prop:sol}. Using the characterization via sequences of subgroups by taking:
    \[
      \begin{aligned}
        \{G_i \cap N\}_{i\in I} \text{ for }& N \\ 
        \left\{\qo{G_i N}{N}\right\}_{i\in J} \text{ for } & \qo{G}{N} 
      \end{aligned}
    \] 

    We give more assertion for the \textbf{third statement}: 

    If \( \{e\} = \qo{G_0}{N} \leq \qo{G_1}{N} \leq \ldots \leq \qo{G_r}{N} = \qo{G}{N}\) as characterization of \( \qo{G}{N} \) being nilpotent, then 
    \[ 
      \{e\} \underbrace{\leq N}_{\text{Since } N\subseteq Z(G)} = G_0 \leq G_1 \leq \ldots \leq G 
    \] 

    satisfies the characterization of \( G \) being nilpotent group.
  \end{proof}

  We now look at what \( p \)-group has.
  \begin{proposition}
    If \( G \) is a \( p \)-group, then \( G \) is nilpotent.
  \end{proposition}

  \begin{proof}
    We proceed by induction on \( \abs{G} \).
    \begin{itemize}
      \item \textbf{Base case}: If \( \abs{G} = p \implies G \cong \; \qo{\ZZ}{p\ZZ} \implies\) nilpotent.
      \item \textbf{Inductive case}: We've proven \( Z(G) \ne \{e\} \) before in \textbf{Proposition} \ref{prop:pg}, it follows that:
        \[
          \abs{\qo{G}{Z(G)}} < \abs{G}
        \] 

        and 
        \[
        \abs{\qo{G}{Z(G)}} \Bigg| \abs{G} \implies \qo{G}{Z(G)} = \{e\} \text{ or a p-group.}
        \] 

        If \( G=Z(G) \) then \( G \) is abelian thus nilpotent. Otherwise we apply induction hypothesis on \( \qo{G}{Z(G)} \) and learn that it is nilpotent. By \textbf{Proposition} \ref{prop:nil}, \( G \) is nilpotent.
    \end{itemize} 
  \end{proof}

  We now give several equivalent definition to nilpotent groups to strengthen the intuition. It is actually strict for some groups to be nilpotent or even solvable.
  \begin{theorem}
    Let \( G \) be a finite group, \( p_1, p_2, \ldots, p_r \) are the primes in the prime factorization of \( \abs{G} \). Let \( P_i \) be the \( p_i \)-Sylow subgroup of \( G \), then the following are equivelent:
    \begin{enumerate}
      \item \( G \) is nilpotent.
      \item For every \( H \lneq G \), we have \( H \subsetneq N_G(H) \).
      \item \( P_i \unlhd G, \; \forall \;i \).
      \item \( G \cong \; P_1 \times \ldots \times P_r \).
    \end{enumerate}
  \end{theorem}

  \begin{lemma}{\textbf{Frattini's Argument}}
    \label{lem:frattini}
    If \( G \) is a finite group with normal subgroup \( H \), and if \( P \) is a Sylow p-subgroup of \( H \), then 
    \[
      G = N_G(P) H 
    \] 

    in particular, by applying it to \( N_G(N_G(P)) \), one can show that
    \[
      N_G(N_G(P)) = N_G(P)
    \] 

    whenever \( G \) is a fnite group and \( P \) is a Sylow \( p \)-subgroup of \( G \).
  \end{lemma}
  \begin{proof}
    The main idea is to show: \( \mathbf{1 \implies 2 \implies 3 \implies 4 \implies 1} \).

    \begin{itemize}
      \item \( \mathbf{(1 \implies 2)} \): Proceed by induction on \( \abs{G} \). If \( \abs{G} = 1 \), things are clear. For the inductive step, If \( Z(G) \not\subseteq H \implies \exists \;a \in Z(G) \backslash H \), and so \( a \in N_G(H) \backslash H\).

        If \( Z(G) \subseteq H \), take \( \overline{G} = \qo{G}{Z(G)} \supsetneq \overline{H} = \qo{H}{Z(G)} \). Since \( G \) is nilpotent, then \( \abs{Z(G)} > 1 \implies \abs{\overline{G}} < \abs{G} \). Note that \( \overline{G} \) is nilpotent by \textbf{Prposition} \ref{prop:nil}. We now apply induction for \( \overline{H} \) and obtain: \( \overline{H} \subsetneq N_{\overline{G}}(\overline{H}) \implies \exists \; aZ(G) \in \qo{H}{Z(G)}, \; a\not\in H \), s.t. \( aZ(G) \overline{H} a^{-1}Z(G) = \overline{H} \implies a H a^{-1} = H\), such completes the induction step.

      \item \( \mathbf{(2\implies 3)} \): By \textbf{Frattini's Argument} \ref{lem:frattini}, \( N_G(N_G(P_1)) = N_G(P_i), \; \forall \; i\), by condition in 2, \( N_G(P_i) = G \iff P_i\unlhd G \). 

      \item \( \mathbf{(3 \implies 4)} \): One shall utilize \textbf{Proposition} \ref{prop:midterm} for this. We show by induction on \( n, \; n \in \br{1,r} \), that:
        \begin{equation}
          \label{eq:nil}
          P_1 \times \cdots \times P_n \to P_1 \cdots P_n \text{ is an isomorphism.}
        \end{equation}

        Note that \( P_1 \cdots P_n \unlhd G \) since \( P_i \unlhd G \; \forall \; i \).

        If this is ok, take \( n=r \), since
        \[
          \abs{P_1 \times \cdots \times P_r} = \abs{G} \implies P_1 \cdots P_r = G 
        \] 

        and yields the result. We now try to proceed the inductive step:

        We have \( P_1 \cdots P_n \unlhd G \) and \( P_{n+1} \unlhd G \) and \( P_1 \cdots P_n \cap P_{n+1} = \{e\}\). By 
        \[
        |P_1 \dots P_n \cap P_{n+1}| \ \Bigg| \ 
        \begin{rcases} 
            |P_1 \dots P_n| = \prod_{i \le n} |P_i| \\
            |P_{n+1}| 
        \end{rcases} \text{relatively prime}
        \] 

        By lagrange theorem and our previous proposition, such holds:
        \[
          \begin{aligned}
            (P_1 \cdots P_n) \times P_{n+1} & \to P_1 \cdots P_n P_{n+1} \text{ is an isomorphism.} \\ 
            (a,b) &\mapsto ab 
          \end{aligned}
        \] 
        And thus:
        \[
          (P_1\times\cdots\timesP_n) \times P_{n+1} \underbrace{\xrightarrow{\sim}}_{\text{ By IH}}(P_1 \cdots P_n) \times P_{n+1} \to P_1 \cdots P_n P_{n+1} \text{ is an isomorphism.} \\ 
        \] 

        The composition is also an isomorphism.

      \item \( \mathbf{(4 \implies 1)} \): Since we know that each \( P_i \) is nilpotent by being a \( p_i \)-group, it is enough to show the following lemma:
        \begin{lemma}{(structrual theorem)}
          If \( G,H \) are nilpotent group, then \( G\times H \) is still nilpotent.
        \end{lemma}

        \begin{proof}
          Check that:
          \[
            Z(G\times H) = Z(G) \times Z(H)
          \] 

          then:
          \[
            \qo{G\times H}{Z(G\times H)} \cong \; \left(\qo{G}{Z(G)}\right) \times \left(\qo{H}{Z(H)}\right)
          \] 

          And apply induction on \( i \):
          \[
            Z_i(G\times H) = Z_i(G) \times Z_i(H) 
          \] 

          and if \( i\gg 0 \):
          \[
            \begin{aligned}
              Z_i(G) &= G \\ 
              Z_i(H) &= H \\ 
              \implies Z_i(G\times H) &= G \times H 
            \end{aligned}
          \] 
        \end{proof}
    \end{itemize}
  \end{proof}
\end{subsection}
