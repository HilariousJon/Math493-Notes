\begin{subsection}{Classification of Finite Simple Groups}
  There exists an \textbf{explicit classification} of finite simple groups!
  \begin{enumerate}
    \item \( \quotient{\ZZ}{p \ZZ} \), \( p \) primes.
    \item \( A_n, \; n\geq 5 \)
    \item \( 12 \) series of  ``finite groups of Lie types".

      For example: \( \text{PSL}_n(\quotient{\ZZ}{p\ZZ}) \) or more generally \( \text{PSL}_n(\FF_q) \).
      \[
        \begin{aligned}
          \text{SL}_n(\quotient{\ZZ}{p\ZZ}) & = \{A \in M_n(\quotient{\ZZ}{p\ZZ}) \; | \; \det A = 1\} \\ 
          \text{PSL}_n(\quotient{\ZZ}{p\ZZ})& = \quotient{\text{SL}_n(\quotient{\ZZ}{p\ZZ})}{\{A = \lambda I_n \; | \; \lambda \in \quotient{\ZZ}{p\ZZ}, \; \lambda^n = 1\}}
        \end{aligned}
      \] 
    \item 26 ``sporaolic examples".
      \[
    \begin{aligned}
      &\text{The largest one: the Monster} \\
      &\quad \text{has order } 2^{46} \cdot 3^{20} \cdot 5^9 \cdot 7^6 \cdot 11^2 \cdot 13^3 \cdot 17 \cdot 19 \cdot 23 \cdot 29 \\
      &\qquad \qquad \quad \cdot 31 \cdot 41 \cdot 47 \cdot 59 \cdot 71 \\
      &\qquad \qquad \quad \sim 8 \cdot 10^{53} \\
      &\text{This can be embedded as a subgroup of } GL_n(\mathbb{C}), \ n = 196,883
    \end{aligned}
      \] 
  \end{enumerate}
\end{subsection}

\begin{section}{Semidirect Product}
  \textbf{General Motivation:} Suppose \( G \) be a group, \( N \unlhd G \), s.t. we understand \( N \) and \( \quotient{G}{N} \), what can we say about \( G \)? In general, not much. However, in particular cases when we have \( G \xrightarrow{\pi} \quotient{G}{N} \) and there is a section of \( \pi \) to be a group homomorphism: \( s: \quotient{G}{N} \to G \), s.t. \( \pi \circ s = Id \), then we can completely describe \( G \) if we know one more piece of data.
  
  The external semidirect product is built by two separate independent group, while the internal semidirect product is to decompose an existing group at hand.
  \begin{subsection}{External Semidirect Product}
    \begin{definition}
    Suppose \( N, H \) be two groups, and we have a group homomorphism
    \[
      \vp: H \to \Aut(N)
    \] 

    We define an operation \( \star \) on \( \mathbf{N\times H} \) by:
    \[
      (n_1, h_1)\star(n_2, h_2) = (n_1\vp_{h_1}(n_2), h_1 h_2)
    \] 

    Such defines a group, denoted as \( N\rtimes H \), which is called the \textbf{external semidirect product} of \( N \) and \( H \) with respect to \( \vp \).
    \end{definition}

    Left for the reader to check that it is indeed a group, checking associativity, existence of identity and existence of inverses are not hard.

    \begin{note}
      \begin{itemize}
        \item If \( \vp: H \to \Aut(N), \; h \mapsto Id \), we recover the usual group structure on \( N\times H \), namely the direct product.
        \item We have a map \( H \to N \rtimes_\vp H, \; h \mapsto (e_N, h) \), this is a group homomorphism, it is clearly injective, with image to be:
          \[
            H'=\{(n,h) \in N \rtimes_\vp H \; | \; n = e_N\}
          \] 

          Similarly, we have \( N \to N \rtimes_\vp H, \; n \mapsto (n,e_H) \), which is also a group homomorphism, injective, with image to be:
          \[
            N'=\{(n,h) \in N \rtimes_\vp H \; | \; h = e_H\}
          \] 

          In fact this is a normal subgroup of \( N\rtimes_\vp H \), can be verified by proving:
          \[
            (n', h')^{-1} \star (n,e) \star (n',h') \in N'
          \] 
        \item We can also prove the following claim by checking:
          \[
            (e_N,h)\star(n,e_H)\star(e_N,h)^{-1} = (\vp_h(n), e_H)
          \] 
          
          \begin{claim}
            via the isomorphism \( N\cong N', \; H\cong H' \), the morphism:
            \[
              \begin{aligned}
                H' &\to \Aut(N') \\ 
                g &\mapsto (n \mapsto gng^{-1})
              \end{aligned}
            \] 

            is given by \( \vp \).
          \end{claim}

        \item We have a group homomorphism:
          \[
            \begin{aligned}
              N\rtimes H &\to H \\ 
              (n,h) &\mapsto h 
            \end{aligned}
          \] 

          This is surjective group homomorphism, with the kernel to be \( N' \cong N \implies \) \( \quotient{N\rtimes_{\vp}H}{N'} \cong H \).

        \item We have a section given by:
          \[
            H \to N \rtimes_\vp H 
          \] 

          \begin{definition}{(\textbf{Section of a Group Extension})}
          \label{def:section}
          Let explicit the short exact sequence:
          \[
              1 \longrightarrow N \xrightarrow{\iota} G \xrightarrow{\pi} H \longrightarrow 1
          \]
          A homomorphism $s: H \to G$ is called a \textbf{section} (or a splitting) of the extension if it satisfies:
          \[
              \pi \circ s = \text{id}_H
          \]
          where $\text{id}_H$ represents the identity map on $H$.

          If such a section exists, we say the extension \textit{splits}, and $G$ is isomorphic to the semidirect product $N \rtimes H$.
          \end{definition}
      \end{itemize}
    \end{note}
  \end{subsection}

  \begin{subsection}{Internal Semidirect Product}
    \begin{definition}
      Suppose \( G \) be any group, and \( H, N \) be subgroups of \( G \) with \( N \unlhd G \) normal. And define:
      \[
        \begin{aligned}
          \vp: H & \to \Aut(N) \\ 
          h &\mapsto \vp_h \\ 
          \vp_h(n) &= hnh^{-1} \qquad \text{ ok since $N \unlhd G$}
        \end{aligned}
      \] 

      We get a map \( N \rtimes_\vp H \xrightarrow{\alpha} G, (n,h) \mapsto nh \). And we claim it is a \textbf{group homomorphism}, with the operation defined as:
      \[
        (n_1, h_1) \star (n_2, h_2) = (n_1 \vp_{h_1}(n_2), h_1 h_2)
      \] 
      
      One should easily verify that.
      
      We define \( G \) to be the \textbf{internal semidirect product} of its subgroups \( N, H \) if \( \alpha \) is an isomorphism.
    \end{definition}
   
    We then give a more straightforward equivalence of this definition.
    \begin{proposition}
      Let \( H, N, G \) be as in the definition, \( G \) is the internal semidirect product of \( N, H \) if and only if:
      \begin{enumerate}
        \item \( N \unlhd G \)
        \item \( G = N \cdot H \)
        \item \( H \cap N = \{e\} \)
      \end{enumerate}
    \end{proposition}

    \begin{proof}
      \( \alpha \) is isomorphism if and only if it is injective and surjective. It is surjective if and only if \( G = N \cdot H \) (This is subgroup since \( N \unlhd G \)). It is injective if and only if:
      \[
        \begin{aligned}
          \ker(\alpha) &= \{(n,h) \; | \; n \in N, \; h \in H, \; nh = e\} \\ 
                       &= \{(n,n^{-1}) \; | \; n \in H \cap N\} \\ 
                       &= \{e\}
        \end{aligned}
      \] 
    \end{proof}
  \end{subsection}
\end{section}
