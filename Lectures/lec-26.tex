\part{Representation Theory} 

\chapter{Group Representations}

Group representation theory is intended to understand group by how it maps to matrices. In general its more kinds of flavor of Category Theory, basically study group by study the morphism between group and general linear transformation.

\begin{section}{Introduction}
  General idea: group representation are linearized version of group actions on sets. 
  \[
    \begin{aligned}
      \sets &\rsa\text{Vector Spaces} \\ 
      \text{Actions by permutations} &\rightsquigarrow \text{Actions by linear isomorphisms}
    \end{aligned}
  \] 

  \begin{definition}{\textbf{(Group Representation)}}
    Let \( G \) be a group, a \underline{representation} of \( G \) is an action of \( G \) on a \textbf{complex} vector space \( V \), such that for all \( g\in G \), the map:
    \[
      \begin{aligned}
        V & \to V \\ 
        v &\mapsto gv 
      \end{aligned}
    \] 

    is linear.
  \end{definition}

  \begin{remark}
    Recall that giving an action of  \( G \) on \( V \) is equivalent to giving a group homomorhism: \( G \xrightarrow{\rho} S_V, \; g \mapsto(v\mapsto gv) \).

    Let \( \GL(V)= \{f: V \to V \; | \; f = \text{ linear isomorphism}\} \subseteq S_V \) be a subgroup.

    Hence giving a group representation is equivalent to \( \Im(\rho) \subseteq \GL(V) \iff \) giving a group homomorphism: \(\rho: G \to \GL(V)  \).
  \end{remark}

  \begin{notation}
    We sometimes denote a representation by \( V \), with the \( G \)-action being understood or by \( \rho \).

    From now on, we will always assume \( \dim_{\CC}(V) < \infty \), this dimension is sometimes called the \textbf{degree of the representation}.
  \end{notation}

  \begin{definition}
    A representation \( \rho: G \to \GL(V) \) is \underline{faithful} if \( \ker(\rho) = \{e\} \).
  \end{definition}

  \begin{remark}
    If \( \dim_{\CC}(V) = n \) and \( e_1, \ldots, e_n \) is a basis of \( V \), then:
    \[
      \begin{aligned}
        \GL_n(V) &\cong \; \GL_n(\CC) \\ 
        (\vp: V \to V) &\to A = (a_{ij})_{i,j\in \br{1,n}} \\ 
        \text{s.t. } \vp(e_j) &= \sum_{i=1}^{n} a_{ij}e_i 
      \end{aligned}
    \] 
  \end{remark}
\end{section}
