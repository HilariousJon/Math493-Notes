\begin{section}{Sylow's Second and Third Theorem}
  Sylow's Second theorem basically tells us how the \( p \)-Sylow subgroup of \( G \) relate to each other, and Sylow's third theorem tells us the possible number of each \( p \)-Sylow subgroup can take. Before diving into the proof, we need to dig more intuition on what group actions can do, i.e. how it \textbf{permute the endian sets}. 
  \begin{subsection}{More on Group Action}
    \begin{eg}
      \label{eg-1}
      Suppose \( G \) is a group acting on a set \( X \):
      \[
        \begin{aligned}
          G \times X & \longrightarrow X \\ 
          (g,x) & \longmapsto gx 
        \end{aligned}
      \] 

      If \( cP(X) \) is the set of all subsets of \( X \), we then get an \textbf{induced group action}:
      \[
        \begin{aligned}
          G \times \cP(X) & \longrightarrow \cP(X) \\ 
          (g, A) & \longmapsto gA = \{ga \; | \; a \in A\}
        \end{aligned}
      \] 

      and its clear that such satisfy the group action properties, we get a group action of \( G \) acting on \( \cP(X) \).

      \textbf{Important exmaple:} If in this case we consider the action on \( G \) acting on itself by conjugation, we can get a correspond action of \( G \) on \( \cP(G) \):
      \[
        (g, A) \longmapsto gAg^{-1} = \{gag^{-1} \; | \; a \in A\}
      \] 

      \begin{note}
        If \( A \) is a subgroup of \( G \), then \( gA g^{-1} \) is also a subgroup of \( G \) \todo{the conjugation map is a group \textbf{automorphism}}. 
      \end{note}
      \end{eg}

      \begin{remark}
        If \( H \) is a \( p \)-Sylow subgroup of \( G \) and \( \sigma: G \to G \) be an automorphism, then \( \sigma(H) \) will also be a \( p \)-Sylow subgroup, in particular, \textbf{every conjugation} of \( H \) which is \( p \)-Sylow subgroup will also be a \( p \)-Sylow subgroup. 
      \end{remark}
  \end{subsection}

  \begin{subsection}{Sylow's Second and Third Theorem}
    \begin{theorem}[\textbf{Sylow's Second Theorem}]
      If \( H \) is a \( p \)-Sylow subgroup of \( G \), \( K \) is \textbf{any \( \mathbf p \)-subgroup} of \( G \), then there exists \( a \in G \), s.t. \( K \subseteq aHa^{-1} \).

      In particular: if \( K \) is a \( p \)-Sylow subgroup, too. Then \( H, K \) are conjugate of each other: \( K = aHa^{-1} \) for some \( a \in G \).
    \end{theorem}

    \begin{theorem}[\textbf{Sylow's Third Theorem}]
      If we denote: 
      \[
        n_p = \# \text{$p$-Sylow subgroups of $G$}
      \] 

      Then we have:
      \[
        \begin{cases}
          n_p \equiv 1 (\bmod p) \\ 
          n_p = (G : N_G(H)) \implies n_p \mid (G:H)
        \end{cases}
      \] 

      with:
      \[
        N_G(H) : =\{g \in G \; | \; gHg^{-1} = H\}
      \] 

      where \( H \) is a \( p \)-Sylow subgroup of \( G \).

      \begin{note}
        \( N_G(H) \) will be the largest subgroup of \( G \) in which \( H \) is a normal subgroup.
      \end{note}

      \begin{note}
        The assertion:
        \[
          n_p = (G : N_G(H)) \implies n_p \mid (G:H)          
        \] 

        is derived from Lagrange's theorem:
        \[
          \begin{aligned}
            H &\leq N_G(H) \leq G \\ 
            \implies (G:H) &= \frac{\abs{G} }{\abs{H} } = \frac{\abs{G} }{\abs{N_G(H)} } \cdot (N_G(H) : H) \\ 
            \text{And } (G : N_G(H)) &= \frac{\abs{G} }{\abs{N_G(H)} }
          \end{aligned}
        \] 
      \end{note}
    \end{theorem}

    We shall prove both of the theorem at the same time.

    \begin{proof}
      We've seen in \textbf{Example} \ref{eg-1}, \( G \) has an action on \( \cP(G) = \{T \; | \; T \leq G\} \) by conjugation: \( (g, T) \to gTg^{-1} \). We now define:
      \[
        \cS = \{H = H_1, H_2, \ldots, H_r\}
      \] 

      which is the collection of the orbits of \( H \) with respect to such action. By \textbf{Orbit-Stablizer Theorem} \ref{orbi-stab}:
      \[
        \begin{aligned}
          r &= (G : Stab_G(H)) \\ 
          Stab_G(H) &= \{g \in G \; | \; gHg^{-1} = H\} = N_G(H) \\ 
          \implies r &= (G : N_G(H))
        \end{aligned}
      \] 
      Now suppose we have a \( p \)-subgroup \( K \) of \( G \), the action of \( G \) on \( \cP(G) \) then induces an action of \( K \) on \( \cP(G) \), and we shall have the canonical injective homomorphism:
      \[
        \quotient{K \cap N_G(H)}{K \cap H} \hookrightarrow \quotient{N_G(H)}{H} 
      \] 

      For the right hand side, see that:
      \[
        \abs{\quotient{N_G(H)}{H}} = \frac{\abs{N_G(H)}}{\abs{H} } \Bigg| \frac{\abs{G} }{\abs{H} } = \frac{\abs{G} }{p^m}
      \] 

      Note that \( \frac{\abs{G}}{p^m} \) is relatively primed to \( p \).

      For the left hand side, it attains a non-negative order (can be \( 0 \)), since \( K \) is a \( p \)-group, it should attain its order to be \textbf{power of $\mathbf p$} by \textbf{Lagrange's Theorem}.

      Now since it is a injective homomorphism, it attains a trival kernel, so by \textbf{Lagrange's theorem + first isomorphism theorem}:
      \[
        \abs{\quotient{K\cap N_G(H)}{K\cap H}} \; \Bigg| \; \abs{\quotient{N_G(H)}{H}}
      \] 

      But notice that the right hand side is coprime with \( p \) but left hand side divides \( p \), this leads to:
      \begin{equation}
        \abs{\quotient{K\cap N_G(H)}{K\cap H}} = 1 \implies K \cap N_G(H) = K \cap H 
        \label{eq-1}
      \end{equation}

      Now suppose \( I \subseteq \{1, \ldots, r\} \) are those such that \( \{H_i \; | \; i \in I\} \) which gives a system of representatives for the orbits of the \( K \) action on \( \cS \), then by class equation:
      \begin{equation}
      r = \abs{\cS} = \sum_{i \in I} (K : \underbrace{Stab_{K}(H_i)}_{= K \cap Stab_G(H_i) = K\cap H_i\text{ by \ref{eq-1}}})
        \label{eq-2}
      \end{equation}
      Now we want to prove:
      \begin{equation}
        (K : K \cap H_i) \iff K \subseteq H_i 
      \end{equation}
      
      If it is not the case, then \( (K : K \cap H_i) \) is divisible by \( p \) since \( K \) is a \( p \)-group. First take \( K = H \), since \( \abs{H} = \abs{H_i}, \; \forall i \), we have \( H \leq H_i \iff i = 1\). Hence by \textbf{Equation} \ref{eq-2}, \( r \equiv 1(\bmod p) \). 

      Suppose that \( K \) is an arbitrary \( p \)-group, if \( K \not \subseteq H_i, \; \forall i \in I \implies p \mid r\) by \textbf{Equation} \ref{eq-2}, which contradicts to \( r \equiv 1 (\bmod p) \). Hence there exists \( i \), s.t. \( K \subseteq H_i = aHa^{-1} \) for some \( a \in G \), such yields \textbf{Sylow's Second Theorem}.

      In particular, we see \( \cS = \{\text{$p$-Sylow subgroups of $G$}\} \implies r = n_p\):
      \[
        \begin{aligned}
          \implies n_p & \equiv 1(\bmod p) \\ 
          n_p &= r = (G : N_G(H))
        \end{aligned}
      \] 

      which yields \textbf{Sylow's Third Theorem}.
   \end{proof}
  \begin{note}
    All \( p \)-Sylow subgroups are conjugate to each other.
  \end{note}
  \begin{remark}
    \[
      n_p = 1 \iff H \unlhd G
    \]
  \end{remark}
  
  We state a small proposition that is helpful when we analyze the group structure along with Sylow's Theorem, it is also the midterm problem of this course.
  \begin{proposition}
    \label{prop:midterm}
    Let \( H, K \unlhd G \), and \( H \cap K = \{e\} \), then:
    \[
      \begin{aligned}
        H \times K & \longrightarrow HK \\ 
        (h,k) & \longmapsto hk 
      \end{aligned}
    \] 

    is a group isomorphism.
  \end{proposition}
  
  We now give two applications of Sylow's Theorem, and give proof to the second one.

  \begin{proposition}
    Let \( G \) be a group such that: \( \abs{G} = pq \) where \( p,q \) are primes, let \( P_p \) and \( P_q \) be two the \( p,q \)-Sylow subgroups of \( G \) respectively, satisfying \( P_p, P_q \unlhd G \), and if \( p < q, \; q \not\equiv 1 (\bmod p) \), we have:
    \[
      G \cong \quotient{\mathbb{Z}}{pq \mathbb{Z}}
    \] 

    and thus \( G \) is \textbf{cyclic} and thus abelian.
    \label{prop:decomp}
  \end{proposition}

  \begin{proposition}
    \label{prop:inherit}
    Suppose \( G \) be a group with order \( 30 = 2 \cdot 3 \cdot 5 \), then:
    \begin{enumerate}
      \item there is a subgroup \( H \leq G \) of order \( 15 \).
      \item \( n_5(G) = 1, \; n_3(G) = 1\).
    \end{enumerate}
  \end{proposition}

  \begin{proof}
    Let \( H \) be a \( 5 \)-Sylow subgroup of \( G \) and \( K \) be a \( 3 \)-Sylow subgroup of \( G \). See that \( \abs{H\cap K} = 1 \) since it has to divide both \( 3 \) and \( 5 \).

    If \( H \unlhd G\), then by \textbf{second isomorphism theorem}, see that \( HK \leq G \) and:
    \[
      \begin{aligned}
        \quotient{HK}{H} &\cong \quotient{K}{H\cap K} \cong K \\ 
        \implies \abs{HK} &= \abs{H} \cdot \abs{K} = 15
      \end{aligned}
    \] 

    Similarly we will get a subgroup of order \( 15 \) if \( K \) is normal.

    Then suppose that both \( H, K \) are not normal subgroups of \( G \). By \textbf{Sylow's Third Theorem}:
    \[
      \left. 
      \begin{aligned}
        n_3(G) &\Bigg| \frac{\abs{G}}{5} = 6 \\ 
        n_3(G) &\equiv 1 (\bmod 5)
      \end{aligned}
      \right\} \implies n_5(G) = 6 \quad (n_5(G) \neq 1 \iff H \ntrianglelefteq G) 
    \] 

    And similarly we have \( n_3(G) = 10 \).

    Notice that the intersection of any \( 2 \) distinct \( 5 \)-Sylow subgroups is the identity, since any of these subgroups is generated by elements that are not the identity. We then have \( 6 \times 4 = 24 \) elements of order \( 5 \) in \( G \), and similarly we get \( 10 \times 2 = 20 \) elements of order \( 3 \). Since \( 24 + 20 > 30 \), such leads to contradtiction $\lightning$.

    Thus we reach to the conclusion that there exists subgroup \( G' \) of \( G \) of order \( 15 \), and since \( (G:G') = 2 \implies G' \unlhd G \). Now since \( \abs{G'} = 3 \times 5 \) and \( 5 \not\equiv 1 (\bmod 3) \), by \textbf{Proposition} \ref{prop:decomp}, \( G' \) is cyclic thus abelian.

    By Sylow's second theorem, \( n_5(G') = n_3(G') = 1 \), and it should deduce that \( n_3(G) = n_5(G) = 1 \):

    Say \( A \) is a \( 5 \)-Sylow subgroup of \( G' \), it should also be a \( 5 \)-Sylow subgroup of \( G \). Now if \( A' \) is another \( 5 \)-Sylow subgroup of \( G \), by Sylow's Second theorem, there exists \( g\in G, \; A' = gAg^{-1} \). Since \( G' \) isi abelian, \( A \unlhd G' \unlhd G \implies A'= gAg^{-1} \subseteq gG'g^{-1} = G' \implies gAg^{-1} \) is also a \( 5 \)-Sylow subgroup of \( G' \) and hence equal to \( A \).
  \end{proof}
  \end{subsection}
\end{section}
