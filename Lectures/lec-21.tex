% This lecture tells about category theory, will be crafted later.
\part{Bonus Lectures}

\chapter{Intro. to Category Theory}

The goal of Catgeory theory is to formalize mathematical objects in a unified way for those notion, constructions, results that appear in different areas.

\begin{section}{Category}
  \begin{definition}[\textbf{Category}]
    A \underline{catgeory} consists of:
    \begin{itemize}
      \item a class \( \ob(\cC) \) of \underline{objects}.
      \item For every two objects \( X, Y \in \ob(\cC) \), we have a set \( \Hom_{\cC}(X,Y) \), which is the \underline{morphisms} in \( \cC \) from \( X \) to \( Y \).
      \item For all \( X, Y, Z \in \ob(\cC) \), a map:
        \[
          \begin{aligned}
            \Hom_{\cC}(X,Y) \times \Hom_{\cC}(Y,Z)& \to \Hom_{\cC}(X,Z) \\ 
            (f,g) &\mapsto g \circ f \quad \text{ ``composition of morphisms"}
          \end{aligned}
        \] 

        s.t.:

        \begin{enumerate}
          \item For all \( X,Y,Z,W \in \ob(\cC) \), we have \( \forall f \in \Hom_{\cC}(X,Y), g\in\Hom_\cC(Y,Z), h\in\Hom_\cC(Z,W) \):
            \[
              X\xrightarrow{f} Y \xrightarrow{g} Z \xrightarrow{h} W 
            \] 

            with associativity:
            \[
              h\circ (g\circ f) = (h\circ g)\circ f 
            \] 

          \item \( \forall X \in \ob(\cC) \), there is \( 1_X \in \Hom_\cC(X,X) \), s.t.
            \[
              \begin{aligned}
                f\circ 1_X &= f \qquad \forall \; f \in \Hom_\cC(X,Y) \\ 
                1_X \circ g &= g \qquad \forall \; g \in \Hom_\cC(Z,X)
              \end{aligned}
            \] 
        \end{enumerate}
    \end{itemize}
  \end{definition}

  \begin{eg}
    \leavevmode 
    \begin{enumerate}
      \item the category of \underline{Sets}:
        \begin{itemize}
          \item objects: all sets.
          \item \( \Hom_{\underline{\sets}}(X,Y) = \{f: X \to Y\} \).
          \item composition: the usual composition of functions.
        \end{itemize}
      \item the category of \underline{Groups}:
        \begin{itemize}
          \item objects: all groups.
          \item morphisms: group homomorphisms.
          \item composition: usual comp. of functions.
        \end{itemize}
      \item the category of \underline{Abelian Groups}:
        \begin{itemize}
          \item objects: all abelian groups.
          \item morphisms: group homomorphisms.
          \item composition: usual comp. of func.
        \end{itemize}
      \item the category of topological spaces \underline{Top}:
        \begin{itemize}
          \item objects: all topological spaces.
          \item morphisms: continuous maps.
          \item composition: usual comp. 
        \end{itemize}
      \item the category of vector spaces over \( \RR \) or \( \CC \), \underline{\( \vect_\RR \)} or \( \underline{\vect_\CC} \):
        \begin{itemize}
          \item objects: all vector spaces over \( \RR \) or \( \CC \).
          \item morphisms: linear maps.
          \item composition: usual comp.
        \end{itemize}
      \item Let \( X \) be given fixed set, the category \( \qo{\underline{\sets}}{X} \):
        \begin{itemize}
          \item objects: \( f: Y \to X \) maps.
          \item morphisms: \( \Hom(Y\xrightarrow{f}X, Y' \xrightarrow{f'} X) = \{g: Y \to Y' \; | \; f'\circ g = f\}\).
            \[
              \begin{tikzcd}
                Y \arrow[rr, dashed] \arrow[rd, "f"] &                               & Y' \arrow[ld, "f'"'] \arrow[ldd, dashed] \\
                                                  & X                             &                      \\
                                                  & Y'' \arrow[u, "f''"] &                     
              \end{tikzcd}
            \] 
          \item composition: usual comp.
        \end{itemize}
      \item Suppose \( (A, \leq) \) be a poset, define a catgeory:
        \begin{itemize}
          \item objects: elements of \( A \).
          \item morphisms:
            \[
              \Hom(x,y) = 
              \begin{cases}
                \{\star\}, \text{ if } x\leq y \\ 
                \emptyset, \text{ otherwise.}
              \end{cases}
            \] 
          \item composition: \( \Hom(x,y) \times \Hom(y,z) \to \Hom(x,z) \) have unique such maps, since \( \leq \) is transitive, also have \( 1_X \; \forall \; x\in A \) since \( \leq \) is reflexive.
        \end{itemize}
    \end{enumerate}
  \end{eg}

  \begin{remark}
    If \( \cC \) is a category, then \( \forall x \in \ob(\cC) \), \( 1_X \) is unique.
  \end{remark}

  \begin{definition}[\textbf{Isomorphism}]
    If \( \cC \) is a category and \( f \in \Hom_\cC(X,Y) \), then \( f \) is an isomorphism if \( \exists \; g \in \Hom_\cC(Y,X) \), s.t.
    \[
      \begin{cases}
        g\circ f = 1_X \\ 
        f\circ g = 1_Y
      \end{cases}
    \] 
  \end{definition}

  \begin{eg}
    \leavevmode 
    \begin{enumerate}
      \item If \( \cC = \underline{\sets} \), then isomorphism is bijection.
      \item If \( \cC = \underline{\text{Gps}} \), then is group isomorphisms.
      \item If \( \cC = \underline{\text{Top}} \), then is homeomorphism.
      \item If \( \cC =\) category associated to \( (A, \leq) \), then isomorphism is:
        \[
          \begin{cases}
            x \leq y \\ 
            y \leq x 
          \end{cases}
        \] 

        which induced a correspondence between the singletons.
    \end{enumerate}
  \end{eg}
\end{section}

\begin{section}{Functors}
  \begin{definition}[\textbf{Functor}]
    If \( \cC \) and \( \cD \) are categories, then a functor:
    \[
      F: \cC \to \cD 
    \] 

    is given by:
    \begin{enumerate}
      \item For every \( X \in \ob(\cC) \), we have \( F(X) \in \ob(\cD) \).
      \item For all \( X, Y \in \ob(\cC) \), we have a map 
        \[
          \Hom_\cC(X,Y) \to \Hom_\cD(F(X), F(Y))
        \] 

        s.t.
        \begin{enumerate}
          \item \( F(1_X) = 1_{F(X)} \quad \forall \; X \in \ob(\cC)\).
          \item \( \forall \; u \in \Hom_\cC(X,Y), v \in \Hom_\cC(Y,Z) \), preserving the composition structures of the morphisms.
            \[
              F(u\circ v) = F(v) \circ F(u)
            \] 
        \end{enumerate}
    \end{enumerate}
  \end{definition}
  
  Functors defines a proper way of jumping between different categories, it behaves well w.r.t. objects, morphisms, and the way of compositions.

  \begin{eg}
    \leavevmode
    \begin{enumerate}
      \item \( F: \underline{\text{Gps}} \to \underline{\sets}\), which is the ``forgetful functors", namely, it is to forget some mathematical structures.
        \begin{itemize}
          \item \( F(G) = G \) where we view \( G \) as sets afterwards.
          \item \( F(f) = f \) where we directly cast of the operation of group.
        \end{itemize}
      \item \( F: \underline{\text{Ab}} \to \underline{\text{Gps}} \)
        \begin{itemize}
          \item \( F(G) = G\).
          \item \( F(f) = f \).
          \item This is more or less like inclusion map, basically do nothing.
        \end{itemize}
      \item \( F: \underline{\Gps} \to \underline{\text{Ab}} \)
        \begin{itemize}
          \item \( F(G) = G^{ab} = \qo{G}{[G,G]} \) with 
            \[
              [G,G] = \angl{\underbrace{[g,h]}_{=ghg^{-1}h^{-1}} \; | \; g,h \in G} 
            \] 

            If \( f: G \to H \) be group homomorphisms, then:
            \[
            \begin{tikzcd}
            G \arrow[r, "f"] \arrow[d, "\pi_G"']           & H \arrow[d, "\pi_H"] \\
            G/[G,G] \arrow[r, dashed, "f^{ab}"] & H/[H,H]
            \end{tikzcd}
            \] 

            Since \( f(ghg^{-1}h^{-1}) = f(g)f(h)f(g)^{-1}f(h)^{-1} \implies f([G,G]) \subseteq [H,H] \). By the universal property of quotient groups, there exists a unique \( f^{ab}: G^{ab} \to H^{ab} \), s.t. \( \pi \circ f = f^{ab} \circ \pi_G \). Explicitly:
            \[
              f^{ab}(g[G,G]) = f(g)[H,H]
            \] 

          \item This satisfy the required condition:
            \begin{itemize}
              \item \( F(1_G) = 1_{G^{ab}} \) is clear.
              \item If:
                \[
                  \begin{tikzcd}
                    G \arrow[r, "f"] \arrow[d] & H \arrow[r, "g"] \arrow[d] & K \arrow[d] \\
                    G^{ab} \arrow[r, "f^{ab}"]  & H^{ab} \arrow[r, "g^{ab}"]  & K^{ab}
                  \end{tikzcd}
                \] 

                By universal property, have \( (g\circ f)^{ab} = g^{ab} \circ f^{ab} \), since we see \( g^{ab} \circ f^{ab} \) also makes the rectangle commutative.
            \end{itemize}
        \end{itemize}
    \end{enumerate}
  \end{eg}

  \begin{proposition}
    If \( F: \cC \to \cD \) is a functor, and \( f\in \Hom_\cC(X,Y) \) which is an isomorphism, then \( F(f) \) is also an isomorphism.
  \end{proposition}

  \begin{eg}
    There is no functor from \( \underline{\Gps} \to \underline{\text{Ab}} \), s.t. it takes \( G \) to \( Z(G) \).
  \end{eg}

  \begin{proof}
    Suppose we have a functor \( F: \underline{\Gps} \to \underline{\text{Ab}}, F(G) = Z(G) \).

    May consider:
    \[
      \angl{(12)} \xhookrightarrow{\alpha} S_3 \xrightarrow{\beta} \qo{S_3}{\angl{(123)}}
    \] 

    the composition is an isomorphism, then:
    \[
      \begin{array}{ccccc}
        F(\langle (12) \rangle) & \xrightarrow{F(\alpha)} & F(S_3) & \xrightarrow{F(\beta)} & F(\qo{S_3}{\langle (123) \rangle}) \\
        \parallel & & \parallel & & \parallel \\
        \langle (12) \rangle & & Z(S_3) = \{e\} & & \qo{S_3}{\langle (123) \rangle}
      \end{array}
    \] 

    on the otherhand, since since the middle group is \( \{e\} \), then the images will be \( \{e\} \), leads to contradiction \( \lightning \).
  \end{proof}

  \begin{eg}
    Let \( \cC \) be arbitrary category, fix \( X \in \ob(\cC) \), define a functor:
    \[
      \begin{aligned}
        \Hom_\cC(X,-): \cC &\to \underline{\sets} \\ 
        Y \in \ob(\cC) & \rsa \Hom_\cC(X,Y) \\ 
        f\in \Hom_\cC(Y,Z) &\rsa \Hom_\cC(X,f) \\
      \end{aligned}
    \] 

    where:
    \[
      \begin{aligned}
        \Hom_\cC(X,f): \Hom_\cC(X,Y) &\to \Hom_\cC(X,Z) \\ 
        g& \mapsto f\circ g 
      \end{aligned}
    \] 
  \end{eg}

  Category theory study the objects by \textbf{How it maps to other objects}.
\end{section}

\begin{section}{Dual of a Category}
  \begin{definition}[\textbf{Dual of a Category}]
    Suppose \( \cC \) is any category, the \underline{dual of \( \cC \)} is the category \( \cC^\circ \) defined by:
    \begin{itemize}
      \item $\ob(\cC^\circ) = \ob(\cC)$.
      \item \( \Hom_{\cC^\circ}(X,Y) = \Hom_\cC(Y,X) \).
      \item composition is the same.
    \end{itemize}
  \end{definition}

  The property of morphisms that compatible to composition is easy to check. Basically the dual is just reversing the morphisms.

  \begin{definition}[\textbf{Monomorphisms}]
    If \( \cC \) is a category and \( f\in \Hom_\cC(X,Y) \), then \( f \) is a \underline{monomorphism} if:
    \[
       \begin{tikzcd}
          Z \arrow[r, "u", shift left=0.5ex] \arrow[r, "v"', shift right=0.5ex] & X \arrow[r, "f"] & Y
       \end{tikzcd}
    \] 
      \( \forall \; u,v \in \Hom_\cC(Z,X) \), s.t. \( f\circ u = f\circ v \implies u=v \).
  \end{definition}

  The \textbf{dual} notion is that of \textbf{epimorphism}.
  \begin{definition}[\textbf{Epimorphism}]
    \( f \in \Hom_\cC(X,Y) \) is \underline{epimorphism} in \( \cC \) if it is a \textbf{monomorphism in \( \mb \cC^\circ \)}. This means:
    \[
        \begin{tikzcd}[column sep=huge, row sep=large]
          X \arrow[r, "f"] & 
          Y \arrow[r, bend left=28, "u"] \arrow[r, bend right=28, "v"'] 
            \arrow[l, dashed, bend left=20]
          & Z 
          \arrow[l, dashed, shift left=0.5ex]
          \arrow[l, dashed, shift right=0.5ex]
        \end{tikzcd}
    \] 

    \( \forall \; u,v \in \Hom_\cC(Y,Z) \), s.t. \( u\circ f = v \circ f \implies u = v \).
  \end{definition}
  
  Usually, Monomorphisms and Epimorphisms corresponds to \textbf{Injective and Surjective} map respectively.
\end{section}
