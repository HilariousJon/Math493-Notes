\begin{section}{Analysis on Decomposition}
  We shall see further application on how to take advantages to the orthogonality of the characters of the irreducible representations.

  \begin{proposition}
    \label{prop:count}
    Given a representation \( V \) and an irreducible representation \( W \), the \# of times \( W \) shows up in a decomposition of \( V \) into irreducible repr is 
    \[
      \angl{\chi_W, \chi_V}
    \] 

    \begin{note}
      This actually gives the proof for the fact that the number is independent on the way of decomposition. 
    \end{note}
  \end{proposition}

  \begin{proof}
    Let \( V \cong \; V_1 \oplus \cdots \oplus V_r \), where \( V_i \) irreducible, then 
    \[
      \chi_V = \sum_{i=1}^{r} \chi_{V_i} 
    \] 

    thus 
    \[
      \angl{\chi_W, \chi _V} = \sum_{i=1}^{r} \angl{\chi _W, \chi _{V_i}} \xlongequal{\text{\textbf{Thm.} \ref{thm:ortho}}} \# \{i \; | \; W \cong \; V_i \} 
    \] 

  \end{proof}

  \begin{proposition}
    Two repr \( V_1, V_2 \) are isomorphic iff \( \chi_{V_1} = \chi_{V_2} \).
  \end{proposition}

  \begin{proof}
    The forward direction is cleared as we've seen in \textbf{Remark} \ref{rem:same-char}. Conversely, followed from \textbf{Proposition} \ref{prop:count} as above: Suppose \( W_1, \ldots, W_r \) are pairwise non-isomorphic irreducible repr of \( G \), s.t. 
    \[
      \begin{rcases}
        V_1 \cong \; \bigoplus_{i=1}^{r} W_i^{\oplus a_i}, \quad V_2 \cong \; \bigoplus_{i=1}^{r}W_i^{\oplus b_i} \\ 
  \text{\textbf{Prop.} \ref{prop:count}} \implies a_i = \angl{\chi_{W_i}, \chi_{V_i}} = \angl{\chi_{W_i}, \chi_{V_2}} = b_i \quad \forall \; i \\ 
      \end{rcases}
      \implies V_1 \cong V_2 
    \] 
  \end{proof}

  This tells us: If we understand the \textbf{characters of a repr.}, then we understand the repr.

  \begin{subsection}{Regular Repr. and Irred. Repr.}
    In this section, we shall use the regular representation \( V^{\text{reg}} \) of \( G \) to deduce information about the irreducible representation of \( G \).

    Recall that the regular representation \( V^{\reg} \) has a basis \( \{e_g \; | \; g \in G\} \) with \( G \) acting via \( h \cdot e_g = e_{hg}, \; \forall \; g,h \in G \). Since \( hg = g \) iff \( h=e \), so 
    \[
      \chi_{V^{\reg}} = 
      \begin{cases}
        0, \text{ if } h \ne e \\ 
        \abs{G}, \text{ if } h = e 
      \end{cases}
    \] 

    Suppose \( V_1, \ldots, V_r \) are \textbf{all} the pairwise non-isomorphic irred. repr. of \( G \), assume:
    \[
      V^{\reg} \cong \; V_1^{\oplus a_1} \oplus \cdots \oplus V_r^{\oplus a_r}
    \] 

    We saw: 
    \[
      \begin{aligned}
        a_i = \angl{\chi_{V_i}, \chi_{V^{\reg}}} &= \frac{1}{\abs{G}} \sum_{g\in G} \chi_{V_i} (g) \overline{\chi_{V^{\reg}}(g)} \\ 
                                                 &= \frac{1}{\abs{G}} \cdot \dim V_i \cdot \abs{G} \\ 
                                                 &= \dim V_i 
      \end{aligned}
    \] 

    Thus we obtain the following proposition.
    \begin{proposition}
      \label{prop:reg}
      If \( W \) is an irred. repr. of \( G \), then the \# of times this appears in a decomposition of \( V^{\reg} \) in irred. repr. is \( \dim W \). 

      In particular, if \( V_1, \ldots, V_r \) are the irred. \( G \)-repr, then:
      \[
        \begin{aligned}
          V^{\reg} &\cong \; V_1^{\dim V_1} \oplus \cdots \oplus V_r^{\dim V_r} \\ 
          \implies \abs{G} &= \sum_{i=1}^{r} \bace{\dim V_i}^2 
        \end{aligned}
      \] 

      \begin{note}
        This is extremely useful for \textbf{determining all the irred. repr.} of a specific \textbf{finite} group.
      \end{note}
    \end{proposition}

    \begin{remark}[\textbf{Decomposition of Characters of Regular Representation}]
      \leavevmode 

      Let \( V_1, \ldots, V_r \) be all the irreducible \( G \)-representations.
      \[
        \begin{aligned}
          \chi_{V^{\reg}}(g) &= \sum_{i=1}^{r} \dim V_i \cdot \chi_{V_i} \\ 
          \implies \sum_{i=1}^{r} \dim V_i \cdot \chi_{V_i}(g) &= 
          \begin{cases}
            0, \quad g \neq e \\ 
            \abs{G}, \quad g = e 
          \end{cases}
        \end{aligned}
      \] 
    \end{remark}

    \begin{eg}
      \( G = S_3 \), it will have the following irred. repr.:
      \begin{itemize}
        \item the trivial \( 1 \)-dim repr.
        \item the repr on \( \{(a_1, a_2, a_3)\; | \; \sum a_i = 0 \} \), which is \( 2 \)-dim. 
        \item the \( 1 \)-dim sign repr. 
      \end{itemize}

      Since \( 1^2 + 1^2 + 2^2 = 6 = \abs{S_3} \), by the proposition, these are \textbf{all} the irreducible repr of \( S_3 \).
    \end{eg}

    \begin{eg}
      \( G = S_4 \), we have known it will have the following repr:
      \begin{itemize}
        \item the trivial \( 1 \)-dim repr. 
        \item the \( 1 \)-dim sign repr.
        \item seen in homework that there are no other \( 1 \)-dim irred. repr. 
      \end{itemize}

      By the \textbf{Proposition} \ref{prop:reg}, there will be remain one \( 2 \)-dim and two \( 3 \)-dim irred. repr. 
    \end{eg}

    \begin{proposition}
      For every \textbf{abelian group} \( G \), all irreducible \( G \)-repr are \( 1 \)-dimensional.

    \begin{fact}
      Given a finite set of endomorphisms \( V \to V \) which are commute and diagonalizable, then there exists a basis of \( V \), s.t. they're all given by diagonal matrices.

      Such statement can be conclude by induction on the size of such sets, and analyzing on the mapping of eigenvectors.
    \end{fact}
    \end{proposition}


    \begin{proof}
      Suppose \( \rho: G \to \GL(V) \) be \( G \)-repr, we've seen \( \rho(g) \) is diagonalizable for all \( g\in G \). One can apply the fact above for \( \{\rho(g) \; | \; g\in G\} \) to get a basis \( e_1, \ldots, e_n \), s.t. each \( \rho(g) \) is given by a diagonal matrix, then each \( \CC e_i \) is a \( G \)-subrepr of \( V \). In particular, if \( V \) is irreducible, it means \( \dim V = 1 \). And by \textbf{Proposition} \ref{prop:reg}, we will have \( \abs{G} \) number of irreducible \( G \)-repr.
    \end{proof}

  \end{subsection}
\end{section}
