\begin{subsection}{Irreducible Representation}
  We now introduce the irreducible representation, which is the building blocks of any representation, it is similar to what simple groups means to finite groups.

  \begin{definition}[\textbf{Irreducible Representation}]
    A \( G \)-repr \( V \) is \underline{irreducible} if 
    \begin{itemize}
      \item \( V \ne \{0\} \).
      \item For every subrepr \( W\subseteq V \), either \( W = \{0\} \) or \( W = V \).
    \end{itemize}
  \end{definition}

  \begin{eg}
    \leavevmode 
    \begin{enumerate}
      \item Every \( 1 \)-dimensional repr is irreducible.
      \item If \( \dim V = 2 \implies \) \( V \) is an irreducible repr iff there exists no common eigenvector for all \( \rho(g), \; g\in groups \).
    \end{enumerate}
  \end{eg}

  \begin{theorem}
    If \( V \) is a repr of the finite group \( G \), then there exists irreducible subrepr \( W_1, \ldots , W_r \). s.t.
    \[
      V = W_1 \oplus \cdots \oplus W_r
    \] 
  \end{theorem}
  \begin{proof}
    \textbf{(Sketch of Proof)}: Use induction on dimension of \( V \), and utilize \textbf{Theorem} \ref{thm:complement}.
  \end{proof}

  The following lemma is very important in terms of \textbf{classifying irreducible representation}, thus we can classify different representation using irreducible representation.

  \begin{lemma}[\textbf{Schur's Lemma}]
    \label{lem:schur}
    Let \( f: V \to W \) be a \textbf{morphism} between \textbf{irreducible} \( G \)-repr, with \( G \) be a finite group. Then
    \begin{enumerate}
      \item If \( V \ncong W \implies f=0\).
      \item If \( V=W \implies f = \lambda \Id_V \) for some \( \lambda \in \CC \).
    \end{enumerate}
  \end{lemma}

  \begin{proof}
    Suppose \( f: V \to W \) is non-zero, then \( \ker(f) \subseteq V \) is a subrepr, and \( f\neq 0 \implies \ker(f) \neq V \). Since \( V \) is irreducible, then \( \ker(f) = \{0\} \implies f \) is injective.

    Let's take \( \Im(f) \subseteq W \) to be subrepr, similarly we see \( \Im(f) = W \implies f \) is surjective. Thus conclude that \( f \) is isomorphism.

    This implies: Suppose now \( V = W \), let \( \lambda \) be an eigenvalue for \( f \), it is an automorphism, so \( \ker(f - \lambda \Id_V) \ne\{0\} \). And clearly \( f-\lambda \Id_V: V\to V \) is morphism of \( G \)-repr.

    Since \( V \) is irreducible, since \( \ker(f-\lambda \Id_V) \ne \{0\} \implies f - \lambda \Id_V = 0\).
  \end{proof}

  We now focus more on what a trivial representation can derive.
  \begin{definition}
    Given ant \( G \)-repr \( V \), define:
    \[
      V^G \coloneqq \{v\in V \; | \; gv = v\} \subseteq V \text{ is a subrepr.}
    \] 
  \end{definition}

  \begin{eg}
    Suppose \( V, W \) are \( G \)-repr, we have:
    \[
      \Hom_\CC(V,W)^G = \{\vp \in \Hom_{\CC}(V,W) \; | \; g\bace{\vp(g^{-1}v)} = \vp(g^{-1}v) \; \forall \; g\in G\}
    \] 

    By definition of morphisms of \( G \)-repr, obtain:
    \[
      \Hom_\CC(V,W)^G = \{f: V\to W \; | \; f \text{ is morphism of $G$-repr.}\}
    \] 
    
    \begin{parenthesis}
      If \( V = V_1 \oplus \cdots \oplus V_r \) where \( V_1,\ldots, V_r \) are \( G \)-repr, \( W \) is a \( G \)-repr, then:
      \[
        \begin{aligned}
          \Hom_\CC(V,W)& \xrightarrow{\sim} \Hom_\CC(V_1, W) \oplus \cdots \oplus \Hom_\CC (V_r, W) \\ 
          \vp & \to (\vp|_{V_1}, \ldots, \vp|_{V_r})
        \end{aligned}
      \] 

      which is an isomorphism by the universal properties of direct sums of \( G \)-repr, thus be a unique decomposition.
    \end{parenthesis}
  \end{eg}

  And we give an application to this example:
  \begin{proposition}
    Suppose \( V = V_1 \oplus \cdots \oplus V_r \) where \( V_i \) are irreducible \( G \)-repr, \( W \) be arbitrary irreducible repr:
    \[
      \#\{i \; | \; V_i \cong \; W\} = \dim_\CC \bace{\Hom_\CC(V,W)^G}
    \] 

    In particular, the LHS is \textbf{independent} of the decomposition of \( V \).
  \end{proposition}

  \begin{proof}
    See that:
    \[
      \begin{aligned}
        \Hom_\CC(V,W) &\cong \; \Hom_\CC(V_1, W) \oplus \cdots \oplus \Hom_\CC(V_r, W) \\ 
        \implies \Hom_\CC(V,W)^G &\cong \; \Hom_\CC(V_1, W)^G \oplus \cdots \oplus \Hom_\CC(V_r, W)^G 
      \end{aligned}
    \] 
    
    Thus we obtain a decomposition of \textbf{morphisms}, so we can apply \textbf{Schur's Lemma} \ref{lem:schur},
    \[
      \Hom_\CC(V_i, W)^G \cong\; 
      \begin{cases}
        0, \; \text{ if } V_i \ncong W \text{ (Contribute no dim.)}\\ 
        \CC, \; \text{ i } V_o \cong W \text{ (Add 1 dim.)}
      \end{cases}
    \] 
  \end{proof}

\end{subsection}
