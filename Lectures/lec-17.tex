\begin{section}{Simple Group}
  ``Simple groups are the basic \textbf{building blocks} of groups". Given a group \( H\unlhd G \), if we understand \( H, \; \quotient{G}{H} \), we can then hope to get some information of \( G \), and thus to decompose such idea, we obtain the concept of \textbf{simple group}.
  \begin{definition}
    A group \( G \), (\textbf{not necessarily finite}), is \underline{simple}. if the following:
    \begin{enumerate}
      \item \( G \ne \{e\} \) 
      \item whenever \( H \unlhd G \), we have either \( H = \{e\} \) or \( H = G \).
    \end{enumerate}

    In particular, there is \textbf{no} interesting normal subgroup of \( G \).
  \end{definition}
  
  We shall give an overview of classifying some good types of groups, but now we already seen that cyclic group and abelian groups give us good enough property. We may ask what will happen to an abelian group if it is also simple.
  \begin{proposition}
    If \( G \) is abelian, then \( G \) is simple \textbf{if and only if} \( G \cong \quotient{\ZZ}{p \ZZ} \).
    \label{prop:abe-simp}
  \end{proposition}

  \begin{proof}
    May assume that \( G \ne \{e\} \), since \( G \) is abelian, it means that every subgroup of \( G \) is normal subgroup. Hence \( G \) is simple if and only if \( G \) has no non-trivial subgroups.

    Equivalently: \( \forall \; x \in G, \; x \ne e \), we have \( G = \angl{x} \). This is ok if \( G \cong \quotient{\ZZ}{p \ZZ} \).

    Conversely: suppose \( G \) satisfying \( G = \angl{x} \), in particular, see that \( G \) is cyclic, so \( G \cong \ZZ \) or \( G \cong \quotient{\ZZ}{n \ZZ}, \; n\geq 2 \). Clearly \( \ZZ \) does not satisfy generated by any element of it: e.g. \( \angl{2} \ne \ZZ \). So \( G \cong \quotient{\ZZ}{n\ZZ}, \; n\geq 2 \). If \( p \) prime, and \( p \mid n \), this implies:
    \[
      \left.
        \begin{aligned}
          \abs{\frac{n}{p}} &= p \\ 
          \angl{\frac{n}{p}} &= G \text{ by assumption}
        \end{aligned}
      \right\} \implies G \cong \quotient{\ZZ}{p\ZZ}
    \] 
  \end{proof}

  We shall give some examples of it.
  \begin{eg}
    \( n\geq 3 \), Dihedral groups, see that:
    \[
      D_{2n} \ne \text{ simple groups}
    \] 

    since
    \[
      \angl{\sigma} \unlhd D_{2n}
    \] 

    because it has index \( 2 \).
  \end{eg}

  \begin{eg}
    $n\geq 3$, \( S_n \) is not simple as \( A_n \unlhd S_n \).
  \end{eg}
  
  \begin{eg}
    If \( p, q \) prime integers, \( \abs{G} = p^2 q \implies G\) is not simple.
  \end{eg}

  \begin{proof}
    \textbf{Sketch of Proof:} Consider whether \( Z(G) = G \), if \( Z(G) = G \), then by \textbf{Proposition} \ref{prop:abe-simp}, it is not simple. If they are not equal, use Sylow's theorem to deduce that it cannot happen that \( n_q \neq 1 \neq n_p \).
  \end{proof}
\end{section}
