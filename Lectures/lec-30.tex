\chapter{Intro. to Multi-linear Algebra}

We basically construct tensor product in this section, the construction of symmetric and exterior product is similar it terms of different universal property.

\begin{section}{Basic Definition}
  \begin{definition}[\textbf{Multi-linear Maps}]
    Given vector spaces \( V_1, \ldots, V_n \), \( W \) and number field \( \KK \), the map:
    \[
      f: V_1 \times \cdots \times V_n \to W 
    \] 

    is \underline{multi-linear} if:
    \[
      f(v_1, \ldots, v_{i-1}, - , v_{i+1}, \ldots, v_n): V_i \to W 
    \] 

    is linear \( \forall \; i \in \br{1,n}\). If \( n=2 \), this is called bilinear.
  \end{definition} 

  \begin{eg}
    \leavevmode
    \[
      \begin{aligned}
        f: \KK^3 \times \KK^2 &\to \KK \\ 
        f((x_1, x_2, x_3), (y_1, y_2)) &= x_1 y_1 - x_2 y_2 
      \end{aligned}
    \] 

    is bilinear.
  \end{eg}

  \begin{definition}[\textbf{Symmetric and Alternating}]
    If \( V,W \) are vector spaces over \( \KK \) and
    \[
      f: \underbrace{V\times\cdots \times V}_{n \text{ times}} \to W 
    \] 

    is multi-linear, then:
    \begin{enumerate}
      \item \( f \) is \underline{symmetric} if:
        \[
          f(v_1, \ldots, v_n) = f(v_{\sigma(1)}, \ldots, v_{\sigma (n)}) \quad \forall \; \sigma \in S_n, \; v_1, \ldots, v_n \in V
        \] 

        Since \( S_n \) is generated by transpositions, it is enough to require:
        \[
          f(v_1, \ldots, v_n) = f(v_1, \ldots, v_{i-1}, v_j, v_{i+1}, \ldots, v_{j-1}, v_i, v_{j+1}, \ldots, v_n) \quad \forall i<j, \; \forall \; v_1, \ldots v_n \in V
        \] 
        \begin{eg}
          Standard scalar product on \( \RR^n \) is symmetric bilinear map:
          \[
            \angl{(x_1, \ldots, x_n), (y_1, \ldots, y_n} \coloneqq \sum_{i=1}^n x_iy_i 
          \] 
        \end{eg}
      \item \( f \) is \underline{alternating} if:
        \[
          f(v_1, \ldots, v_n) = 0 \text{ if } \exists \; i < j \text{ s.t. } v_i = v_j 
        \] 

        or equivalently:
        \[
          \begin{aligned}
            f(v_1, \ldots, v_n) &= -f(v_1, \ldots, v_{i-1}, v_j, v_{i+1}, \ldots, v_{j-1}, v_i, v_{j+1}, \ldots, v_n) \quad \forall i<j, \; \forall \; v_1, \ldots v_n \in V \\ 
            \iff f(v_{\sigma(1)}, \ldots, v_{\sigma (n)}) &= \epsilon (\sigma ) f(v_1, \ldots, v_n)
          \end{aligned}
        \] 

        \begin{eg}
          \leavevmode
          \[
            \begin{aligned}
              f: \KK^n \times \cdots \times \KK^n &\to \KK \\ 
              f(v_1, \ldots, v_n) &= \det \begin{pmatrix}
                v_1 \\ 
                \vdots \\ 
                v_n 
              \end{pmatrix} \; \text{matrix with rows } v_1,\ldots, v_n
            \end{aligned}
          \] 

          This is an alternating multi-linear map
        \end{eg}
    \end{enumerate}
  \end{definition}
\end{section}

\begin{section}{Tensor Product}
  \subsection{Construction}
  \begin{definition}[\textbf{Tensor Product}]
    Let \( V_1, V_2 \) be vector spaces over \( \KK \), a \underline{tensor product} \( V_1 \otimes_{\KK} V_2 \) is a vector space over \( \KK \) together with a bilinear map written:
    \[
      \begin{aligned}
        \vp: V_1 \times V_2 &\to V_1 \otimes_\KK V_2 \\ 
        \vp(v_1, v_2) &= v_1\otimes v_2 
      \end{aligned}
    \] 

    which satisfy the following universal property: For every bilinear map \( f \):
    \[
      \begin{tikzcd}
        V_1 \times V_2 \arrow[r, "f"] \arrow[d, "\varphi"'] & W \\
        V_1 \otimes V_2 \arrow[ur, "g", dashed] & 
      \end{tikzcd}
    \] 

    there exists a \textbf{unique linear map \( \mb{g: V_1 \otimes V_2 \to W} \)}, such that \( f=g\circ\vp \).
  \end{definition}
  \begin{remark}
    This property characterizes \( V\otimes_{\KK} W \) up to a canonical isomorphism:

    if \( V\otimes' W \) is another such tensor product, with \( \psi: V \times W \to V \otimes'W  \) satisfying the same universal property, then there exists a unique linear map \( \alpha \):
    \[
      \begin{tikzcd}[column sep=large, row sep=large]
        V \times W \arrow[r, "\varphi"] \arrow[dr, "\psi"'] & V \otimes W \arrow[d, "\alpha"] \\
        & V \otimes' W
      \end{tikzcd}
    \] 

    s.t. \( \alpha \circ \vp = \psi \) and \( \alpha \) is linear isomorphism. The existence and uniqueness of \( \alpha \) follows from the universal property of \( V\otimes W \), similarly, the universal property of \( V\otimes' W \implies \) there exists a unique linear map \( \beta: V \otimes' W \to V\otimes W \), s.t. \( \beta\circ \psi = \vp \).

    We claim that \( \beta \circ \alpha = 1_{V\otimes W} \), using \( (\beta \circ \alpha) \circ \vp = \beta \circ \psi = \vp \). The uniqueness of the universal property for \( V\otimes W \implies \beta \circ \alpha = 1_{V\otimes W} \), and similarly, universal property of \( V\otimes' W \implies \alpha\circ \beta = 1_{V \otimes' W}\). Thus they are inverse to each other and thus be linear isomorphism.
  \end{remark}

  \begin{proposition}
    Given vector spaces \( V_1, V_2 \) over \( \KK \), there is a tensor product \( V_1 \otimes_{\KK} V_2 \).
  \end{proposition}

  \begin{proof}
    Take \( V \) to be a vector space over \( \KK \) with the basis \( \{e_{(x,y)} \; | \; x\in V_1, y\in V_2\} \).
    \begin{note}
      \( V \) here is a very vast space, infinite dimensional.
    \end{note}

    Take \( W \subseteq V \) to be a linear subspace generated by the following elements:
    \begin{enumerate}
      \item \( e_{(x_1 + x_2,y)} -e_{(x_1,y)} - e_{(x_2,y)}\).
      \item \( e_{(x,y_1 + y_2)} - e_{(x,y_1)} - e_{(x,y_2)} \).
      \item \( e_{(\lambda x,y)} - \lambda e_{(x,y)}\).
      \item \( e_{(x,\lambda y)} - \lambda e_{(x,y)} \).
    \end{enumerate}

    Let \( V_1 \otimes_\KK V_2 := \qo{V}{W} \), and \( \vp \) defined as:
    \[
      \begin{aligned}
        \vp: V_1 \times V_2 &\to V_1 \otimes_\KK V_2 \\ 
        \vp(x,y)&= \overline{e_{(x,y)}}
      \end{aligned}
    \] 
    
    % \begin{note}
    %   As we can see here, the universal property of tensor product is actually derived from the universal property of \textbf{quotient space}.
    % \end{note}
    By the universal property: Suppose \( f: V_1 \times V_2 \to U \) which is bilinear, there exists a unique linear map \( \tilde{g}: V \to U \) such that \( \tilde{g}(e_{(x,y)}) = f(x,y) \; \forall \; x \in V_1, y\in V_2 \).

    We claim that \( W \subseteq \ker(\tilde g) \). This follows because \( f \) is bilinear. For example:
    \[
      \begin{aligned}
        \tilde g(e_{(x_1+x_2,y)} - &e_{(x,y)} - e_{(x_2,u)}) \\ 
        &= f(x_1+x_2,y) - f(x,y) - f(x_2,y) = 0 
      \end{aligned}
    \] 

    Since \( W \subseteq \ker(\tilde g) \), the universal property of quotient \( \qo{V}{W} \) implies there exists a unique linear map:
    \[
      \begin{tikzcd}[column sep=large, row sep=large]
        \qo{V}{W} \arrow[r, "g"] & U \\
        V \arrow[u, "\pi"] \arrow[ur, "{\color{orange}\tilde{g}}"', bend right=10] & & 
      \end{tikzcd}
    \]

    s.t. \( g\circ \pi = \tilde g \). In particular:
    \[
      g(\vp(x,y)) = \tilde g(e_{(x,y)}) = f(x,y) 
    \] 

    This gives the existence of \( g \). One can then check the uniqueness by the definition of \( \tilde g \) and of \( g \), basically using the uniqueness of \( \tilde g \).
  \end{proof}

  One consequence of the construction is that every element in \( V \) is a finite linear combination of \( \sum_{i=1}^n \lambda_i e_{(x_i,y_i)} \), thus every element in \( V_1 \otimes_\KK V_2 \) can be written as:
  \[
    \sum_{i=1}^n \lambda_i \underbrace{(x_i\otimes y_i)}_{\overline{e_{(x_i,y_i)}}} = \sum_{i=1}^n(\lambda_i x_i) \otimes y_i 
  \] 

  \begin{note}
    In general, \( \vp \) is \textbf{not surjective}, but every element in the tensor product \( V_1 \otimes_\KK V_2 \) is a finite sum of elements in \( \im(\vp) \). One shall see that because of the images of a bilinear map, not a linear map, is \textbf{not} a vector space in general.
  \end{note}
  
  \subsection{Functoriality of Tensor Product}
    We introduce that in fact the universal property of tensor product is functorial.

    If \( f_1: V_1 \to W_1 \) and \( f_2: V_2 \to W_2 \) are linear maps, then there exists a unique lienar map denoted:
    \[
      f_1 \otimes f_2: V_1 \otimes V_2 \to W_1 \otimes W_2 
    \] 

    s.t.
    \[
      f_1\otimes f_2(v_1 \otimes v_2) = f_1(v_1) \otimes f_2 (v_2) \quad \forall \; v_1\in V_1, v_2 \in V_2 
    \] 

    Why is that?
    \[
      \begin{tikzcd}[column sep=8em, row sep=5em]
        V_1 \times V_2 \arrow[r, "{(f_1, f_2)}", "\text{linear}"']\arrow[rd , "{\color{blue}\text{bilinear}}", blue, bend left=10] \arrow[d, "\text{bilinear}"', "\varphi_V"] & W_1 \times W_2 \arrow[d, "\text{bilinear}", "\varphi_W"'] \\
        V_1 \otimes V_2 \arrow[r, "f_1 \otimes f_2", dashed]  & W_1 \otimes W_2
      \end{tikzcd}
    \] 

    the composition is still bilinear as shown. This is by the universal property of \( V_1 \otimes V_2 \implies \) there exists unique linear map \( f_1 \otimes f_2 \), such that the diagram is commutative.

    This functorial in the following senses:
    \begin{enumerate}
      \item If \( W_1 = V_1, W_2 = V_2, f_1 = \Id_{V_1}, f_2 = \Id_{V_2} \implies f_1 \otimes f_2 = \Id_{V_1 \otimes V_2} \) by using the uniqueness in the definition. 
      \item If we also have \( g_1: W_1 \to U_1, g_2: W_2 \to U_2 \) are linear maps, then:
        \[
          (g_1 \circ f_1) \otimes (g_2 \circ f_2) = (g_1 \otimes g_2) \circ (f_1 \otimes f_2)
        \] 

        in short, the following diagram is commutative:
        \[
        \begin{tikzcd}[column sep=large, row sep=large]
          V_1 \times V_2 \arrow[r, "{(f_1, f_2)}"] \arrow[d] & W_1 \times W_2 \arrow[r, "{(g_1, g_2)}"] \arrow[d] & U_1 \times U_2 \arrow[d] \\
          V_1 \otimes V_2 \arrow[r, "f_1 \otimes f_2"'] & W_1 \otimes W_2 \arrow[r, "g_1 \otimes g_2"'] & U_1 \otimes U_2
        \end{tikzcd}
        \] 

        the left square and the right square are commutative, so the rectangle is commutative.
    \end{enumerate}

    \begin{proposition}
      If \( e_1, \ldots, e_n \) is a basis for \( V_1 \) and \( f_1, \ldots, f_m \) is a basis for \( V_2 \), then:
      \[
        \{e_i \otimes f_j \; | \; i \in br{1,n}, j \in \br{1,m}\}
      \] 

      is a basis for \( V_1 \otimes V_2 \).

      In particular, \( \dim(V_1 \otimes V_2)  = \dim V_1 \cdot \dim V_2 \).
    \end{proposition}

    \begin{proof}
      
    \end{proof}
\end{section}
