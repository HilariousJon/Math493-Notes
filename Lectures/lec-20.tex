\chapter{Classification of Finite Groups}

Finite group theory is already a well-studied field, people are already been able to classify different types of groups in finite order with good or relatively good property, and decompose the finite group into the building blocks, namely the simple groups. We've studied simple groups before, so in this chapter we shall introduce how we will decompose the finite groups into simple factors through \textbf{Jordan-H\"older theorem}. We will then introduce some special groups with relatively good properties, which are built by abelian groups. A small glimpse of it will be:
\[
  \text{cyclic} \subseteq \text{abelian} \subseteq{nilpotent} \subseteq \text{solvable}
\] 

and we will also try to dig some more property of \( p \)-group.

\begin{section}{Composition Series and Jordan-H\"older Theorem}
  \begin{definition}[\textbf{Composition Series}]
    Given a group \( G \), a \underline{composition series} of \( G \) is given by a sequence of subgroups:
    \[
      \{e\} \leq N_r \leq N_{r-1} \leq \ldots \leq N_0 = G 
    \] 

    such that \( \forall \; i \in [r-1] \):
    \begin{enumerate}
      \item \( N_{i+1} \unlhd N_i \) 
      \item \( \quotient{N_i}{N_{i+1}} \) are simple groups 
    \end{enumerate}

    We call \( r \) to be the \underline{length} of the series, and the \( \quotient{N_i}{N_{i+1}} \) to be the \underline{simple factors} in the series 

    \begin{note}
      Some trivial case worth noting:
      \begin{itemize}
        \item \( r=0 \iff G = \{e\} \) 
        \item \( r = 1 \iff G \) is simple
      \end{itemize}
    \end{note}
  \end{definition}
  
  \begin{definition}
    If \( G \) has a composition series, then \( G \) has finite length. 
  \end{definition}

  \begin{proposition}
    Every finite group has finite length 
  \end{proposition}

  \begin{proof}
    \textbf{Sketch of proof}: proof by induction on \( \abs{G} \), and \textbf{glue the composition series} of \( N \) and \( \quotient{G}{N} \) if there exists \( N \unlhd G \).
  \end{proof}

  \begin{theorem}[\textbf{Jordan-H\"older Theorem}]
    Given two composition series for \( G \), denoted as:
    \[
      \begin{aligned}
        \{e\} &= N_r \leq \ldots \leq N_1 \leq N_0 = G \\
        \{e\} &= N_s' \leq \ldots \leq N_1' \leq N_0' = G \\
      \end{aligned}
    \]

    Then:
    \begin{itemize}
      \item \( r = s \)
      \item and 
        \[
          \begin{aligned}
          \quotient{N_{r-1}}{N_r} &, \ldots, \quotient{N_0}{N_1} \\ 
          \quotient{N'_{s-1}}{N'_s} &, \ldots, \quotient{N'_0}{N'_1}
          \end{aligned}
        \] 

        are pairwise isomorphism, after possibly reordering.
        \begin{note}
          This is basically stating that the \textbf{simple factors} are the \textbf{building blocks}.
        \end{note}
    \end{itemize}
  \end{theorem}
  
  We need a lemma to proof the original theorem.
  \begin{lemma}
    \label{lem:comp}
    If \( G \) has a composition series, \( N \unlhd G \), then \( N \) has a composition series.
  \end{lemma}

  \begin{proof}{(\textbf{Proof of Lemma \ref{lem:comp}})}
    Suppose we start with a composition series of \( G \), and consider for \( N \):
    \[
      \{e\} = N \cap G_r \leq \ldots \leq N \cap G_1 \leq N \cap G_0 = N 
    \] 

    Now what can we say about the following injective map?
    \[
      \quotient{N\cap G_i}{N\cap G_{i+1}} \hookleftarrow \quotient{G_i}{G_{i+1}} 
    \] 

    By normality of \( N \) in \( G \), see:
    \[
      \quotient{N\cap G_i}{N\cap G_{i+1}} \subseteq \quotient{G_i}{G_{i+1}}
    \] 

    With the image of the map also normal in \( \quotient{G_i}{G_{i+1}} \), and since the simple factor is simple:
    \begin{itemize}
      \item either \( \quotient{N\cap G_i}{N\cap G_{i+1}} = \{e\} \)
      \item or \( \quotient{N\cap G_i}{N\cap G_{i+1}} = \quotient{G_i}{G_{i+1}} \)
    \end{itemize}

    Then \textbf{After removing repeated factors, i.e. Those} \( \mathbf{\{e\}} \), we get a composition series of \( N \).
  \end{proof}
  \begin{proof}{(\textbf{Proof of Jordan-H\"older Theorem})}
    We shall proceed the proof by induction on the shortest length of a composition series of \( G \), in particular, our aim is to see that if \( G \) has a composition series with length \( r \), then all composition series of \( G \) should have length \( r \) and the pairwise isomorphism statement holds.
    \begin{itemize}
      \item \textbf{Base case}: If this is \( 0 \): \( G = \{e\} \), which is trivial, if this is \( 1 \), then \( G \) is simple, every composition series should have length to be \( 1 \), also trivial.
      \item \textbf{Inductive case}: Suppose \( r\leq s \), and we assume by Induction we know the theorem for those groups that admit a composition series with length \( r-1 \), then there are two cases:
        \begin{itemize}
          \item \( N_1 = N'_1 \): we then get a composition series for those two subgroups, and we can apply induction hypothesis see that \( r-1 = s-1 \) and the pairwise isomorphism statement also holds and call a day.
          \item \( N_1 \ne N'_1 \): We then look at what we can say about \( N_1 N'_1 \)?

            First note that \( N_1 \not\subseteq N'_1 \), otherwise:
            \[
              \{e\} \ne \quotient{N'_1}{N_1} \unlhd \quotient{G}{N_1} \; \textbf{(Simple)} \quad \lightning
            \] 

            Similarly, \( N'_1 \not\subseteq N_1 \).

            Then \( N_1 \subseteq N_1 N'_1 \unlhd G \), since both \( N_1, N'_1 \) are normal subgroups.

            Since \( \quotient{G}{N} \) is simple, then either \( N = N_1 N'_1 \implies N'_1 \subseteq N_1 \; \lightning \) or \( N_1 N'_1 = G \), clearly the latter case.

            Hence \( G = N_1 N'_1 \), then by the second isomorphism theorem:
            \[
              \quotient{G}{N_1} = \quotient{N_1 N'_1}{N_1} \cong \quotient{N'_1}{N_1 \cap N'_1} \implies \quotient{G}{N'_1} \cong \quotient{N_1}{N_1 \cap N'_1}
            \] 

            We then want to use a composition series of \( N_1 \cap N'_1 \) to apply induction.

            \[
              \begin{tikzcd}
                  & N_1 \arrow[rd, hook] & \\
                  N_1 \cap N_1' \arrow[ru, hook] \arrow[rd, hook] & & G = N_1 N_1' \\
                  & N_1' \arrow[ru, hook] & 
              \end{tikzcd}
            \] 

            and by the second isomorphism theorem:
            \begin{equation}
              \label{eq:2iso}
              \begin{aligned}
                \quotient{N_1}{N_1 \cap N'_1} &\cong \quotient{G}{N'_1} \\ 
                \quotient{N'_1}{N_1 \cap N'_1} & \cong \quotient{G}{N_1}
              \end{aligned}
            \end{equation}

            And in particular, groups on the \textbf{LHS} are \textbf{simple}.

            Since \( N_1 \cap N'_1 \subseteq G\) and \( G \) has a composition series, then by \textbf{Lemma} \ref{lem:comp}, \( N_1 \cap N'_1 \) also has a composition series, and in particular \( N_1 \cap N'_1 \unlhd G\).

            We now choose one such composition series:
            \[
              \{e\} = N''_t \leq \ldots \leq N''_{1} \leq N''_0 = N_1 \cap N'_1
            \] 

            Then we obtain four kinds of composition series for \( G \):
            \begin{enumerate}
              \item \( \{e\} = N_r \leq \ldots \leq N_2 \leq N_1 \leq G \) 
              \item \( \{e\} = N''_t \leq \ldots \leq N''_1 \leq N_1 \cap N'_1 \leq N_1 \leq G\) 
              \item \( \{e\} = N''_t \leq \ldots \leq N''_1 \leq N_1 \cap N'_1 \leq N'_1 \leq G\) 
              \item \( \{e\} = N'_s \leq \ldots \leq N'_2 \leq N'_1 \leq G\)
            \end{enumerate}

            Since \( N_1 \) has a composition series of length \( r-1 \), by \textbf{IH}, the \textbf{first and second} series satisfy the condition of the theorem.

            By \textbf{Formula} \ref{eq:2iso}, the \textbf{second and third} series satisfy the condition in the theorem.

            By \( N'_1 \) has a composition series of length \( r-1 \), by \textbf{IH}, the \textbf{third and fourth} series satisfy the condition in the theorem.

            Thus the \textbf{first and fourth} series satisfy the condition of the theorem.
        \end{itemize}
    \end{itemize}
  \end{proof}

  \begin{remark}
    If \( G \) being finite, and:
    \[
      \{e\} = G_r \leq \ldots \leq G_1 \leq G = G_0 \; \text{is a composition series.}
    \] 
    
    Then induction on \( i \) + Lagrange theorem gives us:
    \[
      \begin{aligned}
        \abs{G_i} &= \prod_{j=i}^{r-1} \abs{\quotient{G_j}{G_{j+1}}} \\ 
        \text{by } \abs{G_i} &= \abs{G_{i+1}} \cdot \abs{\quotient{G_i}{G_{i+1}}} \\ 
        \implies \abs{G} &= \prod_{i=0}^{r-1} \abs{\quotient{G_i}{G_{i+1}}}
      \end{aligned}
    \] 
  \end{remark}

  \begin{remark}
    If \( G \) is abelian and finite, then since any of its subgroup will be normal, then given a composition series denoted as above, we have:
    \[
      \abs{\quotient{G_i}{G_{i+1}}} = \text{ prime integer}
    \] 

    In fact we will see later this is an \textbf{equivalent} definition for a group to be \textbf{nilpotent}.
  \end{remark}
  \begin{proof}
    \textbf{Sketch of Proof}: This directly yields by doing prime factorization on \( \abs{G} \) and matching the size of the simple factor on the prime number and combined with \textbf{Proposition} \ref{prop:decomp}.
  \end{proof}
  
  \begin{eg}
    Both \( \quotient{\ZZ}{4\ZZ} \) and \( \quotient{\ZZ}{2\ZZ} \times \quotient{\ZZ}{2\ZZ} \) have simple factors \( \quotient{\ZZ}{2\ZZ} \) twice, in particular this tells us simple factors aare not unique properties to a group, it is only related to \( \abs{G} \), \textbf{just like the prime numbers to a particular integers}.
  \end{eg}
\end{section}
