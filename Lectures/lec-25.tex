\begin{subsection}{Presentations of Groups by Generators and Relations}
  The question is raised naturally from the universal property of free groups.

  Given a group \( G \) and a subset \( S \subseteq G \). By the universal property of \( F(S) \), there exists a unique group homomorphism
  \[
    \begin{aligned}
      f: F(S) & \to G \\ 
      f|_S &= \text{ the inclusion map}
    \end{aligned}
  \] 

  Since \( \Im(f) = \angl{S} \) since \( F(S) \) is generated by \( S \), in particular, if \( S \) generates \( G \), \( f \) is surjective.

  The question is: \textbf{How to describe the kernel of it?} If we can describe its kernel, we can construct isomorphism through isomorphism theorem.
  \begin{definition}[\textbf{Normal Closure}]
    Given a group \( G \) and a subset \( A \), the \underline{normal closure} of \( A \) is the \textbf{smallest normal subgroup} of \( G \) containing \( A \).
    \[
      \bigcap_{A \subseteq H, H\leq G} H 
    \] 
    
    Also one can show that the normal closure of \( A \) is 
    \[
      \angl{\{gag^{-1} \; | \; g \in G, \; a\in A\}}
    \] 
  \end{definition}

  Given \( A \), if \( N \) is the normal closure of \( A \), may consider \( \qo{G}{N} \) is the ``largest" quotient of \( G \) in which the elements of \( A \) is \textbf{identity}.

  \begin{definition}[\textbf{Presentation}]
    A \underline{presentation} of a group \( G \) is a pair \( (S, \fR) \) where \( S \) is a set, \( \fR \subseteq F(S) \), such that if \( N \) is the normal closure of \( \fR \) in \( F(S) \), then:
    \[
      \qo{F(S)}{N} \cong \; G 
    \] 
  \end{definition}

  \begin{intuition}
    \( S \) as generator and \( \fR \) as the relation, so there is no redundant relation by the normal closure, thus we have an isomorphism. 
  \end{intuition}

  \begin{eg}
    If \( n \geq 3 \), then \( D_{2n} \) has a presentation with 
    \[
      \begin{aligned}
        S &= \{\sigma, \tau\} \\ 
        \fR &= \{\sigma^n, \tau^2, \tau\sigma^{-(n-1)}\tau\sigma \} 
      \end{aligned}
    \] 
    usually written as:
    \[
      D_{2n} = \{\sigma, \tau \; | \; \sigma^n = e, \; \tau^2 = e, \; \tau \sigma = \sigma ^{n-1}\tau \}
    \] 
  \end{eg}

  \begin{proof}
    Let \( N \) be the normal closure of \( \fR\subseteq F(\{\tau, \sigma\}) \), it is clear:
    \[
      \begin{aligned}
        \text{if } F(\{\tau ,\sigma \}) &\overset{\varphi}{\twoheadrightarrow} D_{2n} \\ 
        \text{and } \fR &\subseteq \ker(\vp) \\ 
        \implies N &\subseteq \ker(\vp)
      \end{aligned}
    \] 

    We get a group homomorphism which is also surjective:
    \[
      \qo{F(\{\tau ,\sigma \})}{N} \overset{\overline{\vp}}{\twoheadrightarrow} D_{2n} 
    \] 

    By the relation \( \tau\sigma^{-(n-1)}\tau\sigma \), every element in \(\qo{F(\{\tau ,\sigma \})}{N}  \) will be in the form \( \sigma^i\tau^j \). By the relation \( \sigma^n, \; \tau^2 \), may assume \( i \in \br{0,n-1}, \; j \in\br{0,1} \), then we have \( \leq 2n \) elements in \( \qo{F(\{\tau ,\sigma \})}{N} \), then it has to be injection and thus bijection, thus be an isomorphism.
  \end{proof}

  \begin{definition}
    A group \( G \) is finitely presented if there exists presentation \( (S, \fR) \) of \( G \) with both \( \fR, S \) finite.
    \begin{note}
      \leavevmode 
      \begin{enumerate}
        \item In general, finite generated \( \not\implies \) finite presented.
        \item Know that given \( (S_1, \fR_1) \), \( (S_2, \fR_2) \), figuring out whether:
          \[
            \qo{F(S_1)}{N_1} \cong \; \qo{F(S_2)}{N_2}
          \] 

          is \textbf{undecidable}. (very hard, no algorithm to do)
      \end{enumerate}
    \end{note}
  \end{definition}

  \begin{theorem}
    Every finite group is finitely presented.
  \end{theorem}

  \begin{proof}
    Let \( G \) be finite group, take \( S = G \) and \( \fR = \{g_1g_2g_3^{-1} \in F(G) \; | \; g_3 = g_1 g_2 , \; g_1,g_2,g_3 \in G\} \)

    \begin{note}
      Note that \( g_1g_2g_3^{-1} \) is in \( F(G) \) not \( G \), so there is really no inverse cancellation. The relation here is intended to \textbf{go through all multiplication combination}, and for each pair of \( (g_i,g_j) \), we have \( g_k = g_i \cdot g_j \) in \( G \), and thus we construct a word by such three \textbf{alphabet} as \( g_i g_j g^{-1}_k \).
    \end{note}

    May consider the map:
    \[
      \begin{aligned}
        \vp: F(G) & \to G \\ 
        \vp|_G &= Id \; (g_i \mapsto g_i)
      \end{aligned}
    \] 

    such is clearly surjective, with the left hand side is the alphabet and the right hand side is the corresponding group element.

    By construction, clear that \( \fR \subseteq \ker(\vp) \), let \( N \) be normal closure of \( \fR \) in \( F(G) \), then \( N \subseteq \ker(\vp) \), so 
    \[
      \exists ! \; \overline{\vp}: \; \qo{F(G)}{N} \to G 
    \] 

    such that:
    \[
      \begin{tikzcd}
          F(S) \arrow[r, "\pi", two heads] \arrow[d, "\varphi"'] & F(S)/N \arrow[dl, "\bar{\varphi}", dashed] \\
          G & 
      \end{tikzcd}
    \]
    
    by the universal property of quotient. 

    Now let \( A = \{\pi(g) \; | \; g \in G\} \), see that:
    \[
      \begin{rcases}
        \angl{A} = \qo{F(G)}{N} \text{ since } \angl{g \; | \; g\in G} = F(g) \\ 
        \underbrace{A \text{ is \textbf{closed} under multiplication and inverses}}_{A \text{ itself is a group, so it equals to its \textbf{generator}}}
      \end{rcases}
      \implies A = \qo{F(G)}{N} 
    \] 

    thus:
    \[
      \begin{rcases}
        \abs{\qo{F(G)}{N}} \leq \abs{G} \\ 
        \overline{\vp} \text{ surjective} 
      \end{rcases}
      \implies \overline{\vp} \text{ is isomorphism.}
    \] 
  \end{proof}
\end{subsection}
