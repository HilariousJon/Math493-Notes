\chapter{Group Actions}

\begin{section}{Introduction}
  We now lay our focus to group actions, group actions are useful because we can endowed the \textbf{symmetric structure} of a group into other mathematical objects through group actions, specifically:
  \begin{itemize}
    \item often groups acts on various mathematical structure, such as sets, topological spaces, manifolds, etc.
    \item It will be of great significance for us to consider the actions of a group on itself via \textbf{conjugation}.
  \end{itemize}
  
  \begin{definition}
    Let's fix a group $G$ and a set $X$, an \textbf{action} (say also a left action of ) $G$ on $X$ is a map:
    \[
      \begin{aligned}
        G \times X &\to X \\
        (g,x) &\mapsto gx \\
      \end{aligned}
    \]

    such that the following holds:
    \[
      ex = x \quad \forall \; x \in X 
    \] 
    \[
      g(hx) = (gh) x \quad \forall \; g, h \in G, \; x \in X
    \] 
  \end{definition}

  We now introduce an \textbf{equivalent formulation} for group action:

  Recall that:
  \[
    S_X = (\{\text{bijections } X \to X\}, \circ)
  \] 

  is a group.
  
  \begin{definition}
    Now suppose we have the action of $G$ on $X$ as above, we may define a map $\varphi: G \to S_X$ as follows:
    for every $g \in G$, $\varphi(g)$ which written as $\varphi_g$ is the map:
    \[
      \varphi_g :X \to X, \varphi_g (x) = gx
    \] 
  \end{definition}
  
  It is easy to see that by inheritance of the existance of inverses in $G$, $\varphi_g$ is a bijection. In particular, one can see that it is actually a \textbf{group homomorphism}.
  
  And the following conclusion is easy to deduce:
  \begin{conclusion}
    \[
      \text{\textbf{\{Actions of $G$ on $X$}\}} \leftrightarrow \text{\{\textbf{Group Homomorphism} $G \to S_X$\}}
    \] 

    forms a \textbf{bijection}.
  \end{conclusion}

  We then give some examples of group actions:
  \begin{eg}
    Given any set $X$, we have the identity, \textbf{trivial} group action given by the group homomorphism:
    \[
      S_X \xrightarrow{Id} S_X
    \] 
    
    which is equivalent to the action of $S_X$ on $X$ by:
    \[
      S_X \times X \to X \; , \; (f,x) \mapsto f(x)
    \] 
  \end{eg}

  \begin{eg}
    If $n > 3$ and $P_n$ be the regular $n$-gon, we then have a group homomorphism:
    \[
      D_{2n} \to S_{P_n}
    \] 

    which leads to an action of $D_{2n}$ on \( P_n \)

    \begin{note}
      See that in this case \( D_{2n} \) preserve the distance structure within the regular \( n \)-gon.
    \end{note}
  \end{eg}

  \begin{eg}
    The group \( GL_n(\mathbb C) \) acts on \( \mathbb C^n \) via:
    \[
      (A, u = \begin{pmatrix}
        u_1 \\ 
        \vdots \\ 
        u_n
      \end{pmatrix}) \mapsto Au
    \] 

    which represent the matrix as \textbf{linear transformation}. Such corresponds to the group homomorphism:
    \[
      \begin{aligned}
        & GL_n(\mathbb C) \to S_{\mathbb C^n} \\ 
        A & \mapsto \text{corresponds linear transformation on } \mathbb C^n
      \end{aligned}
    \] 
  \end{eg}

  \begin{eg}{(Cayley's Theorem)}
    Define an action of \( G \) on itself by:
    \[
    \begin{aligned}
      G\times G &\to G \\ 
      (g,h) & \mapsto g \cdot h
    \end{aligned}
    \]

    which acts by the natural left multiplication. Note such corresponds to a group homomorphism:
    \[
      G \xrightarrow{\varphi} S_G
    \] 

    And we shall have:
    \begin{proposition}{(Cayley)}
      \( \varphi \) is always injective
    \end{proposition}

    In particular, if \( G \) is finite, \( G \) is then \textbf{isomorphic} to a subgroup of \( S_n \).
    \[
      G \cong \Im(\varphi) \subseteq S_G 
    \] 
    
    The proof is immediate by showing \( \ker (\varphi) = \{e\} \) by cancellation.
  \end{eg}

  \begin{eg}
    Suppose \( H\leq G \), we have:
    \[
    \begin{aligned}
      G \times (G / H)_l & \to (G / H)_l \\ 
      (g, ah) & \mapsto ga H \\ 
    \end{aligned}
    \]

    easy to see such is a group action after checking well-definedness. Such action is induced by the action of group on itself, note \( H \) here is \textbf{not necessarily normal}.
  \end{eg}

  \begin{eg}{(Group action by \textbf{Conjugation})}
    The following will be the most interesting example for us. First recall we have an \textbf{automorphism} given by \( g \; \in G \):
    \[
      \alpha_g: G \to G \; , \; \alpha_g(x) = g x g^{-1}
    \] 

    Moreover, observe \( Aut(G) \leq S_G \), so we have a group homomorphism:
    \[
      \begin{aligned}
        G & \to Aut(G) \leq S_G \\ 
        g & \mapsto \alpha_g
      \end{aligned}
    \] 

    We can understand \( Aut(G) \) as those \textbf{permutation that preserve the group structure}.

    In particular, by our discussion, we get an action of \( G \) on itself:

    \[
      (g,x) \mapsto gxg^{-1}
    \] 
  \end{eg}
  
\end{section}
