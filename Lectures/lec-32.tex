\section{Orthogonality of Characters}
  We would gradually define the Hermitian product of characters in this section, notably, there is orthonormality within.

  Consider \( \Fun(G) = \{f: G\to \CC\} \) be arbitrary functions, this is a vector space over \( \CC \):
  \[
    \begin{aligned}
      (f+g)(x) &= f(x) + g(x) \; \forall \; x \in G \\ 
      (\lambda f)(x) &= \lambda (f(x)) \; \forall \; x \in G 
    \end{aligned}
  \] 

  A basis of \( \Fun(G) \) is given by \( \{f_x \; | \; x \in G\} \), with:
  \[
    f_x(y) = 
    \begin{cases}
      1 \text{ if } x = y \\ 
      0 \text{ otherwise}
    \end{cases}
  \] 
  
  \begin{note}
    \( h \in \Fun(G), \; h = \sum_{x\in G}h(x) \cdot f_x\).
  \end{note}

  Thus we see
  \[
    \dim(\Fun(G)) = \abs{G}
  \] 

  \begin{definition}[\textbf{Hermintian Product over $\mathbf{Fun(G)}$}]
    We have on \( \Fun(G) \), the \underline{Hermitian product} \( \angl{-,-} \) defined by:
    \[
      \angl{\vp, \psi} \coloneqq \frac{1}{\abs{G}} \sum_{g\in G}\vp(g) \overline{\psi(g)} \in \CC
    \] 

    such that:
    \begin{enumerate}
      \item \( \angl{-, \psi}: \Fun(G) \to \CC \) is linear.
      \item \( \angl{\vp, \psi} = \overline{\angl{\psi, \vp}} \).
      \item \( \angl{\vp, \vp} > 0\), if \( \vp \ne 0 \), it is \( \frac{1}{\abs{G}} \sum_{g\in G} \abs{\vp(g)}^2 \).
    \end{enumerate}
  \end{definition}

  \begin{theorem}
    \label{thm:ortho}
    \leavevmode
    \begin{enumerate}
      \item If \( \rho_1, \rho_2 \) are \textbf{non-isomorphic irreducible} repr of \( G \), then 
        \[
          \angl{\chi_{\rho_1}, \chi_{\rho_2}} = 0 
        \] 
      \item If \( \rho_1 \cong \; \rho_2 \) are \textbf{irreducible}, then 
        \[
          \angl{\chi_{\rho_1}, \chi_{\rho_2}} = 1
        \] 
    \end{enumerate}
  \end{theorem}

  \begin{proof}
    \leavevmode
    \begin{idea}
      Starting with any linear map \( \vp: \CC^n \to \CC^m \) use the ``averaging procedure" from the proof of \textbf{Theorem} \ref{thm:complement} to get from \( \vp \) a morphism of \( G \)-repr \( \psi: \CC^n \to \CC^m \), then by \textbf{Schur's Lemma} \ref{lem:schur}, \( \psi = 0 \) in case 1 and \( \psi = \lambda \Id \) in case 2. Choose \( \vp \) suitably and taking the trace shall yield the results.
    \end{idea}
    
    May assume: 
    \[
      \begin{aligned}
        \rho_1 : G & \to \GL_n(\CC) \\ 
        \rho_2: G& \to \GL_n(\CC) 
      \end{aligned}
    \] 
    In the setting of 2, may assume \( \rho_1 = \rho_2\) since by \textbf{Remark} \ref{rem:same-char}, isomorphic repr have the same character.
    Given any linear map \( \vp: \CC^n \to \CC^m \), take \( \psi \) given by
    \[
      \psi(v) = \sum_{g\in G}g^{-1}\vp(gv)
    \] 

    the \textbf{key point} is: this is a morphism of \( G \)-repr:
    \[
      \psi(hv) = \sum_{g\in G} \underbrace{g^{-1}}_{h(gh)^{-1}}\vp(ghv) = h \sum_{g\in G}(g^{-1} \vp(gv)) = h \vp(v)
    \] 

    \begin{note}
      \leavevmode 
      \begin{enumerate}
        \item \( gv \) here is \( \rho_1(g) v\), the group action for \( v\in \CC^n \). 
        \item \( g^{-1} \) here is \( \rho_2(g^{-1}) \), the group action for \( \Im(\vp) \in \CC^m \).
      \end{enumerate}
    \end{note}

    If \( \vp \) described w.r.t. the standard basis of \( \CC^n, \CC^m \) by the matrix \( (\gamma_{ij})_{\substack{1 \le i \le m \\ 1 \le j \le n}} \) and 
    \[
      \begin{aligned}
        \rho_2(g^{-1}) &\longleftrightarrow (b_{pq}(g))_{1\leq p,q\leq m} \\ 
        \rho_1(g) &\longleftrightarrow (a_{kl}(g))_{1\leq k,l\leq n}
      \end{aligned}
    \] 
    
    then \( \psi \) correspond to th matrix 
    \begin{equation}
      \label{eq:gam-mat}
      \sum_{g\in G} (b_{pq}(g))\cdot(\gamma_{ij})\cdot(a_{kl}(g))
    \end{equation}

    \begin{itemize}
      \item Suppose we are in \textbf{Case 1}: By Schur's Lemma, \( \psi = 0 \). Now fix \( i\leq m, j\leq n \) and take 
        \[
        \gamma_{i'j'} =
        \begin{cases}
          1, \text{ if } i'=i, j'=j \\ 
          0, \text{ otherwise}
        \end{cases}
      \]

      Now \textbf{Formula} \ref{eq:gam-mat} is \( 0 \implies \forall \; p \leq m, \forall \;l\leq n \) :
      \[
        \sum_{g\in G} b_{pi}(g) a_{jl}(g) = 0 
      \] 

      Take \( p=i, l=j \implies \sum_{g\in G} b_{ii}(g)a_{jj}(g) = 0\), take \( \sum_{i,j} \) on both sides:
      \[
        \sum_{g\in G} \underbrace{\Trace\bace{\rho_2(g^{-1})}}_{\chi_{\rho_2}(g^{-1}) = \overline{\chi_{\rho_2}(g)}} \underbrace{\Trace\bace{\rho_1(g)}}_{\chi_{\rho_1(g)}} = 0 \implies \angl{\chi_{\rho_1}, \chi_{\rho_2}} = 0 
      \] 

      \item Suppose we are in \textbf{Case 2}: Then \( \rho_1 = \rho_2 \) are irreducible repr, by \textbf{Schur's Lemma} \ref{lem:schur}, \( \forall \; \vp = (\gamma_{ij}) \), we have \(\psi = \lambda(\vp)\Id \) for some \( \lambda (\vp) \). 

        For every fixed \( i,j\leq n \), let
        \[
          \gamma_{i'j'} = 
          \begin{cases}
            1, \text{ if } i=i', j=j' \\ 
            0, \text{ otherwise}
          \end{cases}
        \] 

        For all \( p,l \leq n \), we have 
        \[
          \sum_{g\in G} b_{pi }(g) a_{jl}(g) = \lambda_{i,j} \delta_{p,l} \quad \text{ for some } \lambda_{i,j} \in \CC 
        \] 

        Take \( p=i, l=j \implies \) 
        \begin{equation}
          \label{eq:sec}
          \sum_{g\in G}b_{ii}(g)a_{jj}(g) = \lambda_{i,j}\delta_{i,j}
        \end{equation}

        Take sum \( \sum_{i,j} \) on both sides, obtain:
        \[
          \underbrace{\sum_{g\in G}\Trace\bace{\rho_2(g^{-1})}\Trace\bace{\rho_1(g)}}_{\abs{G} \cdot \angl{\chi_{\rho_1}, \chi_{\rho_2}}} = \sum_{i=1}^{n} \lambda_{i,i} 
        \] 
        
        Recall that \( \rho_1 = \rho_2 \implies b_{ij} = (a_{kl})^{-1} \), take \( i=j \) in \textbf{Formula} \ref{eq:sec} and take the sum \( \sum_{i=1}^{n} \), we have:
        \[
          \begin{aligned}
            \sum_{g\in G} \delta_{p,l} &= \sum_{i=1}^{n}\lambda_{i,i}\delta_{p,l} \\ 
            \text{Let } p=l \implies \sum\lambda_{i,i}&= \abs{G} \\ 
            \implies \abs{G} \cdot \angl{\chi_{\rho_1} , \chi_{\rho_2}} &= \abs{G} \\ 
            \implies \angl{\chi_{\rho_1} , \chi_{\rho_2}} &= 1 
          \end{aligned}
        \] 
    \end{itemize}
  \end{proof}
  
  Thus we see the \textbf{irreducible representations} yields a \textbf{orthonormal system}.

  \begin{corollary}
    There are only finitely many classes of \( G \)-repr if \( G \) be finite group.
  \end{corollary}

  \begin{proof}
    If \( \rho_1, \ldots, \rho_n \) are pairwise non-isomorphic irreducible representation, the theorem implies:
    \[
      \angl{\chi_{\rho_i}, \chi_{\rho_j}} = \delta_{i,j}
    \] 

    This implies \( \chi_{\rho_1}, \ldots, \chi_{\rho_n} \) are linearly independent, and \( n \) will have an upperbound, which is 
    \[ 
      \dim\bace{\Fun(G)} = \abs{G}
    \]
  \end{proof}

  \begin{corollary}
    A \( G \)-repr \( \rho \) is irreducible if and only if \( \chi_{\rho}, \chi_{\rho} = 1 \).
  \end{corollary}

  \begin{proof}
    Consider a decomposition into irreducible representation 
    \[
      \rho \cong\; \bigoplus_{i=1}^{r} V_{i}^{\oplus a_i}, \quad V_i \ncong V_j\; \text{ if } i\ne j 
    \] 

    then 
    \[
      \chi_{\rho} = \sum_{i=1}^{r} a_i \chi_{V_i} 
    \] 

    The theorem tells us \( \angl{\chi_{\rho}, \chi_{\rho}} = \sum_{i=1}^{r} a_i^2 \), hence \( \angl{\chi_{\rho}, \chi_{\rho}} = 1\) iff \( r=1 \) and \( a=1 \).
  \end{proof}

